\documentclass[]{MastersDoctoralThesis} % The class file specifying the document structure

\usepackage{arabtex}
\usepackage[utf8]{inputenc} %for international characters
\usepackage{utf8}
\setcode{utf8}
\usepackage{tcolorbox} % For creating boxes

\newtcolorbox{arabicbox}{
  colback=white, % Background color
  colframe=white, % Border color
  boxrule=1pt, % Border thickness
  arc=0pt, % Corner radius
  right=0cm, % Space on the right
  left=0cm, % Space on the left
  top=5mm, % Space at the top
  bottom=5mm, % Space at the bottom
  fontupper=\normal, % Font size for the text
}

\usepackage{xspace}
\newcommand{\Ar}[1]{{\scriptsize \<#1>\xspace}}
\newcommand{\TrAr}[1]{\arabtrue\transfalse {\scriptsize \Ar{#1}} \arabfalse\transtrue \RL{#1}\arabtrue\transfalse} 
 


\usepackage[T1]{fontenc} % Output font encoding for international characters
\usepackage{mathpazo} % Use the Palatino font by default
\usepackage{hyperref}
\usepackage{amsmath}
\usepackage{amsfonts}
\usepackage{amssymb}
\usepackage{graphicx}
\usepackage{caption}
\captionsetup{justification = raggedright, singlelinecheck = false}
\usepackage{subcaption}
\usepackage{tikz}
\usepackage{multirow}
\usepackage{xcolor}
\usepackage{graphicx}
\usepackage{subcaption}
\usepackage[bottom]{footmisc}
\usepackage{float}
\usepackage{pgfplots}
\pgfplotsset{compat=1.18}
\usepackage{setspace}\linespread{1.3}
\usepackage{pdflscape}
\usepackage{times} %Sabbah: font ACL used
\usepackage{microtype} %Sabbah: line spaces
\usepackage{tcolorbox} % For creating boxes


%Sanad
\usepackage{babel}
%----------- Generate syntax highlighted code (optional)--------
%\usepackage{verbatim}
%\usepackage{minted}
%Copy the following PdfLatex build command (without%) in order to run minted:
%TexStudio > Preferences > Commands > PdfLaTeX
%pdflatex -synctex=1 -interaction=nonstopmode --shell-escape %.tex
%% Debug for Python user: if you encounter pygmentize error, do this:
%Copying pygmentize file from "/Library/Frameworks/Python.framework/Versions/2.6/bin/pygmentize" to "/usr/local/bin"
%See more: https://www.overleaf.com/learn/latex/Code_Highlighting_with_minted

%----------- Use .bib to generate reference list -------------
% \usepackage[backend=bibtex,style=authoryear,natbib=true]
\usepackage[backend=bibtex,style=aclnatbib,natbib=true]{}

%{biblatex}
%\usepackage{natbib}
% Use the bibtex backend with the authoryear citation style (which resembles APA)
%Commands: 
%use \textcite{} to cite as Author(year)
%use \parencite{} to cite as (Author, year)
%\usepackage[autostyle=true]{csquotes} % Required to generate language-dependent quotes in the bibliography
%\addbibresource{reference.bib} % The filename of the bibliography

%---------- Style Options (uncomment to enable) -----------
%\renewcommand*{\thefootnote}{(\arabic{footnote})} %Enable to Put braces around footnotes
%\addto\captionsenglish{\renewcommand\chaptername{Section}} %Enable to Use "sections" instead of "chapters"
%\newcommand{\RomanNum}[1]{\MakeUppercase{\romannumeral #1}} %Enable this option to be able to write Roman Numerals (I, II, III) by calling an equivalent number.
%usage: \RomanNum{10} %will generate "X"

%----------------------------------------------------------------------------------------
%	MARGIN SETTINGS
%----------------------------------------------------------------------------------------

\geometry{
	paper=a4paper, % Change to letterpaper for US letter
	inner=2.5cm, % Inner margin
	outer=3.8cm, % Outer margin
	bindingoffset=.5cm, % Binding offset
	top=1.5cm, % Top margin
	bottom=1.5cm, % Bottom margin
	%showframe, % Uncomment to show how the type block is set on the page
}

%------------------------
%	THESIS INFORMATION
%------------------------
\usepackage{adjustbox}
\usepackage{lscape}
\usepackage{rotating}
\usepackage{xspace}
\newcommand{\etal}{{\em et al.}\xspace}
\newcommand{\BfPara}[1]{\vspace{0.1mm}\noindent{\bf #1.}\xspace}

\newcommand{\Quote}[2][3cm]{%
  {\setlength{\parindent}{0pt}% Disable the default paragraph indentation
  \raggedleft
  \hspace*{#1}%  horizontal space to push the first line to the right
  \hangindent=#1% Set the hanging indentation for subsequent lines
  \textit{#2}% Make the text italic
  \par % Ensure proper paragraph termination
  \vspace{1cm}
  
  }
}

\newcommand{\IfPara}[1]{\vspace{0.1mm}\noindent{\em #1.}\xspace}
\usepackage{pifont}
\thesistitle{Concept Drift and Model Forgetting in Machine Learning-Based Android Malware Detection } % Your thesis title, this is used in the title and abstract, print it elsewhere with \ttitle

\supervisor{Prof. David Mohaisen}
% Your supervisor's name, this is used in the title page, print it elsewhere with \supname

\examiner{examiner1} % Your examiner's name, this is not currently used anywhere in the template, print it elsewhere with \examname

\degree{Doctor of Philosophy Proposal} % Your degree name, this is used in the title page and abstract, print it elsewhere with \degreename

\author{Ahmed Sabbah} % Your name, this is used in the title page and abstract, print it elsewhere with \authorname

\addresses{Earth} % Your address, this is not currently used anywhere in the template, print it elsewhere with \addressname

\subject{Computer Science} % Your subject area, this is not currently used anywhere in the template, print it elsewhere with \subjectname
\keywords{} % Keywords for your thesis, this is not currently used anywhere in the template, print it elsewhere with \keywordnames

\university{\href{https://www.birzeit.edu/en}{Birzeit University}} % Your university's name and URL, this is used in the title page and abstract, print it elsewhere with \univname
\department{PhD Program in Computer Science} % Your department's name and URL, this is used in the title page and abstract, print it elsewhere with \deptname\

\group{BZU Cyber Security} % Your research group's name and URL, this is used in the title page, print it elsewhere with \groupname

\faculty{Faculty of Graduate Studies and Research} % Your faculty's name and URL, this is used in the title page and abstract, print it elsewhere with \facname

\AtBeginDocument{
\hypersetup{pdftitle=\ttitle} % Set the PDF's title to your title
\hypersetup{pdfauthor=\authorname} % Set the PDF's author to your name
\hypersetup{pdfkeywords=\keywordnames} % Set the PDF's keywords to your keywords
}

\begin{document}


\frontmatter % Use roman page numbering style (i, ii, iii, iv...) for the pre-content pages

\pagestyle{plain} % Default to the plain heading style until the thesis style is called for the body content

%----------------------------------------------------------------------------------------
%	TITLE PAGE
%----------------------------------------------------------------------------------------

\begin{titlepage}
\begin{center}
%\vspace*{.05\textheight}
 %{\scshape\LARGE \univname\par}\vspace{1.5cm} % University name

\includegraphics[scale=0.15]{Logo.png} \vspace{0.5cm} %Download the logo from here: https://i.imgur.com/gRUcrqO.png

%\textsc{ Doctoral Thesis Proposal }\\[0.1cm] % Thesis type




{\large \bfseries \facname \\ \deptname \par}\vspace{1.7cm} % Thesis title

{\large \textit{Dissertation Proposal}}\\[0.3cm] % University requirement text}

\HRule \\[0.1cm] % Horizontal line
{\LARGE \bfseries \ttitle\par} % Thesis title
\HRule \\[1.5cm] % Horizontal line

     
\begin{minipage}[t]{\textwidth}

\begin{center}
    % \begin{flushleft} \large
    {\Large \authorname} \\
   
\end{center}

% Author name - remove the \href bracket to remove the link
% \end{flushleft}
% \end{minipage}
% \begin{minipage}[t]{0.4\textwidth}
% \begin{flushright} \large
% \emph{Supervisor:} \\
% {\supname}\\
% Radi% Supervisor name - remove the \href bracket to remove the link  
% \end{flushright}
\end{minipage}\\[2cm]
 

Supervisor: Prof. David Mohaisen\\
Co-Supervisors: Prof. Samer Al-Zein, Prof. Radi Jarrar 
\vfill  \vfill \vfill


\begin{quote}
\begin{center}
\textit{A detailed proposal submitted in partial fulfillment of the degree of Doctor of Philosophy in Computer Science at Birzeit University}.
\end{center}
\end{quote}

\vfill

{\large \today}

%\includegraphics{logo} % University/department logo - uncomment to place it
 
%\vfill
\end{center}
\end{titlepage}

\newpage
\thispagestyle{empty}
\mbox{}



\begin{Committee}
\begin{center}

{\Large \bfseries \ttitle\par}\vspace{0.7cm} % Thesis titlet the PDF's author to 
 {\Large \bfseries \authorname\par}\vspace{3cm}
\end{center} 


{\Large \textbf{Dissertation Committee:}}\\
\begin{itemize}  % Reduce the left margin of itemize

\item Prof. David Mohaisen, Birzeit University (Chair) \\
\item Prof. Samer Al-Zein, Birzeit University\\
\item Prof. Name , Affiliation \\
\item Prof. Name , Affiliation \\
\item Prof. Name , Affiliation \\

\end{itemize}
  
\end{Committee}

%\iffalse %Uncomment this and the command \fi below to disable this page
%----------------------------------------------------------------------------------------
%	DECLARATION PAGE
%----------------------------------------------------------------------------------------

\begin{declaration}
%\addchaptertocentry{\authorshipname} % Add the declaration to the table of contents
\vspace{1cm}
\noindent I, \authorname, declare that this thesis titled, "{\ttitle}" and the work presented in it are my own. I confirm that:

\begin{itemize} 
\item I hereby declare that the dissertation/proposal I am submitting is entirely my own work
except where otherwise indicated.

\item I hereby declare that the content of my thesis is entirely my own work and has not been generated or written using artificial intelligence tools. Any assistance provided by AI tools was limited solely to language editing, such as improving grammar, spelling, or formatting.


%\item This work was done wholly or mainly while in candidature for a research degree at this University.
%\item Where any part of this thesis has previously been submitted for a degree or any other qualification at this University or any other institution, this has been clearly stated.
%\item Where I have consulted the published work of others, this is always clearly attributed.
%\item Where I have quoted from the work of others, the source is always given. With the exception of such quotations, this thesis is entirely my own work.
\item I have acknowledged all main sources of help.
\item Where the thesis is based on work done by myself jointly with others, I have made clear exactly what was done by others and what I have contributed myself.\\
\end{itemize}
 
\noindent Signed:\\
\rule[0.5em]{25em}{0.5pt} % This prints a line for the signature
 
\noindent Date:\\
\rule[0.5em]{25em}{0.5pt} % This prints a line to write the date
\end{declaration}


\newpage
\thispagestyle{empty}
\mbox{}

% \fi %uncomment, together with \Leftrightarrowalse above to disable pages
\cleardoublepage

%----------------------------------------------------------------------------------------
%	QUOTATION PAGE
%----------------------------------------------------------------------------------------

\vspace*{0.2\textheight}


\noindent{\itshape Thanks to my solid academic training, today I can write hundreds of words on virtually any topic without possessing a shred of information, which is how I got a good job in journalism.}\bigbreak


\newpage
\thispagestyle{empty}
\mbox{}


%----------------------------------------------------------------------------------------
%	ACKNOWLEDGEMENTS
%----------------------------------------------------------------------------------------


\begin{acknowledgements}
	%\addchaptertocentry{\acknowledgementname} % Add the acknowledgements to the table of contents
I would like to express my thanks to \dots OR 
I wish to express my sincere gratitude to \dots

Kindly acknowledge the support you received from each individual or organization that contributed to your research. The idea is not simply a gesture of gratitude (to say thank you), but this is the place where you should specify the contributions made by others.

Make sure to recognize the input of your supervisor(s), comments and feedback from the thesis committee, colleagues, technical support staff, those who provided datasets, and anyone else who played a role in 
	
\end{acknowledgements}

%%%%%%%%%%%% Empty page for printing purpose


%----------------------------------------------------------------------------------------
%	ABSTRACT PAGE
%----------------------------------------------------------------------------------------
% English Abstract
% \selectlanguage{english}
% \renewcommand{\abstractname}{Abstract} % Set English title
\begin{abstract}

\addchaptertocentry{\abstractname} % Add the abstract to the table of contents
The Arabic word (\Ar{مرحبا}) is translated can be written and translated at the same time as (\TrAr{مرحبا}).
In recent years, mobile malware detection has become a prominent topic in the field of cybersecurity. Machine learning models dominate this field with a variety of detection approaches. The accumulation of machine learning models in malware detection has led to significant advances, with some models achieving accuracy rates as high as 0.99. However, researchers faced the issue of concept drift phenomenon in which statistical properties or defining features of the target variable change over time in an unpredictable manner. In the context of Android malware, this phenomenon is characterized by a lack of knowledge about the underlying causes. However, the solutions that have been proposed to detect and mitigate concept drift by updating the model to increase the ability to detect new malware may result in forgetting models where the model becomes aging and cannot detect old samples. In this research, we use both quantitative and qualitative methods to investigate concepts drift in Android malware detection models using machine learning and deep learning techniques. This approach will provide an understanding of the factors that cause concept drift, the methods to detect it, retraining strategies, and addressing model forgetting in the context of malware detection. The first phase of this research will be a case study, through the analysis of different evolving malware families, including changes in their behavior or code (static or dynamic features) that lead to concept drift. The second phase is a comparative analysis, by comparing different versions of malware families and investigating the changes in their features that result in concept drift and identifying which specific features are most susceptible to changes over time and result in concepts drift. In addition, we will focus on establishing robust evaluation metrics and continuous monitoring strategies to evaluate the performance of machine learning models used to detect Android malware, mainly in unbalanced distributions in datasets. This perspective is particularly valuable for developing practical solutions and strategies to mitigate this phenomenon, taking into account the various contexts and limitations faced by researchers in this domain. Additionally, we will investigate model forgetting and how to preserve the balance between concept drift and model forgetting. Additionally, we will investigate model forgetting and how to preserve the balance between concept drift and model forgetting.Additionally, we will investigate model forgetting and how to preserve the balance between concept drift and model forgetting.Additionally, we will investigate model forgetting and how to preserve the balance between concept drift and model forgetting.Additionally, we will investigate model forgetting and how to preserve the balance between concept drift and model forgetting. Additionally, we will investigate model forgetting and how to preserve the balance between concept drift and model forgetting.Additionally, we will investigate model forgetting and how to preserve the balance between concept drift and model forgetting.Additionally, we will investigate model forgetting and how to preserve the balance between concept drift and model forgetting. Additionally, we will investigate model forgetting and how to preserve the balance between concept drift and model forgetting.Additionally, we will investigate model forgetting and how to preserve the balance between concept drift and model forgetting.Additionally, we will investigate model forgetting and how to preserve the balance between concept drift and model forgetting.
\vfill
\keywordnames
\end{abstract}
\
\newpage

%----------------------------------------------------------------------------------------
%	Arabic ABSTRACT PAGE
%----------------------------------------------------------------------------------------

% Arabic Abstract
% \selectlanguage{arabic}
% \renewcommand{\abstractname}{ \Ar{\Large ملخص}} % Change to Arabic title

\begin{otherlanguage}{arabic}
\vspace*{100pt}
\renewcommand{\abstractname}{ \Ar{\Large ملخص}}
\begin{abstract}

%\begin{arabicbox}
\begin{arabtext}

أقلمة النماذج (\LR{domain adaptation}) حسب المجال هي إحدى تقنيات التعلم العميق في الذكاء الاصطناعي التي تهدف إلى تكييف نموذج مدرب مسبقًا على مجال معين لاستخدامه في مجال آخر. على سبيل المثال، يمكن تدريب نموذج آلي لتصنيف وجهات نظر المستخدمين في مجال السياحة، ثم تكييف هذا النموذج لإعادة استخدامه في تصنيف وجهات النظر في مجال السياسة. تهدف أقلمة النماذج وتكييفها إلى تقليل تكلفة تجميع وتصنيف بيانات تدريبية في مجالات مختلفة نظرًا لارتفاع تكلفة تصنيف وتوسيم بيانات جديدة لكل مجال. يتسع  مفهوم المجال ليعني حقلًا أو موضوعًا مثل الاقتصاد أو السياسة أو الفن، وقد تعني لهجة أو لغة. على سبيل المثال، يمكن تكييف نموذج آلي تم تدريبه باستخدام نصوص عامية فلسطينية لإعادة استخدامه مع نصوص العامية المصرية. وفي أحيان أخرى، قد تعني كلمة "مجال" أزمنة مختلفة. مثلًا ، يمكن تدريب نموذج على بيانات في مجال الزراعة جُمعت عام 1920 ومن ثم أقلمة وتكييف هذا النموذج لاستخدامه لفهم بيانات زراعية عام 2020. ازدادت أهمية هذا الموضوع كثيرًا  في السنوات الأخيرة بسبب انتشار التعلم الآلي والحاجة إلى بيانات تدريب  في الكثير من المجالات. هناك عدة طرق ومنهجيات لأقلمة النماذج، مثل طرق تغيير تمثيل البيانات ، خوارزميات التعلم الذاتي ، خوارزميات تدريس المعرفة والتعلم التبايني  وغيرها.


أقلمة النماذج (\LR{domain adaptation}) حسب المجال هي إحدى تقنيات التعلم العميق في الذكاء الاصطناعي التي تهدف إلى تكييف نموذج مدرب مسبقًا على مجال معين لاستخدامه في مجال آخر. على سبيل المثال، يمكن تدريب نموذج آلي لتصنيف وجهات نظر المستخدمين في مجال السياحة، ثم تكييف هذا النموذج لإعادة استخدامه في تصنيف وجهات النظر في مجال السياسة. تهدف أقلمة النماذج وتكييفها إلى تقليل تكلفة تجميع وتصنيف بيانات تدريبية في مجالات مختلفة نظرًا لارتفاع تكلفة تصنيف وتوسيم بيانات جديدة لكل مجال. يتسع  مفهوم المجال ليعني حقلًا أو موضوعًا مثل الاقتصاد أو السياسة أو الفن، وقد تعني لهجة أو لغة. على سبيل المثال، يمكن تكييف نموذج آلي تم تدريبه باستخدام نصوص عامية فلسطينية لإعادة استخدامه مع نصوص العامية المصرية. وفي أحيان أخرى، قد تعني كلمة "مجال" أزمنة مختلفة. مثلًا ، يمكن تدريب نموذج على بيانات في مجال الزراعة جُمعت عام 1920 ومن ثم أقلمة وتكييف هذا النموذج لاستخدامه لفهم بيانات زراعية عام 2020. ازدادت أهمية هذا الموضوع كثيرًا  في السنوات الأخيرة بسبب انتشار التعلم الآلي والحاجة إلى بيانات تدريب  في الكثير من المجالات. هناك عدة طرق ومنهجيات لأقلمة النماذج، مثل طرق تغيير تمثيل البيانات ، خوارزميات التعلم الذاتي ، خوارزميات تدريس المعرفة والتعلم التبايني  وغيرها.

\end{arabtext}
%\end{arabicbox}
\end{abstract}
\end{otherlanguage}

% \selectlanguage{english}

%----------------------------------------------------------------------------------------
%	LIST OF CONTENTS/FIGURES/TABLES PAGES
%----------------------------------------------------------------------------------------
\tableofcontents % Prints the main table of contents
%\listoffigures % Prints the list of figures
%\listoftables % Prints the list of tables

%----------------------------------------------------------------------------------------
%	ABBREVIATIONS
%----------------------------------------------------------------------------------------

\begin{abbreviations}{ll} % Include a list of abbreviations (a table of two columns)
\textbf{DT} & Decision Tree  \\
\textbf{LR} & Logistic Regression  \\
\textbf{KNN} & K-Nearest Neighbor \\
\textbf{BN} &  Bayesian Network  \\
\textbf{SVM} &  Support Vector Machines   \\
\textbf{BOW} & Bags-of-Words  \\
\textbf{AD} & AdaBoost  \\
\textbf{NN} & Neural Networks  \\
\textbf{NB} & Naive Bayes  \\
\textbf{SMO} &  Sequential Minimal Optimization  \\
\textbf{SEL} &   Ensemble Stacking   \\
\textbf{CNN} &  Convolutional Neural Networks  \\
\textbf{GRU} &  Gated Recurrent Unit \\
\textbf{EXT} &  Extremely Randomized Trees \\
\textbf{Stacking} &  Multiple base classifiers to make a final prediction  \\
\textbf{VF2} &  Algorithm is used to find the most significant common subgraph \\
           & between samples of each malware family.  \\
\textbf{XGBoost} & Extreme Gradient Boosting  \\
\textbf{SMOTE} & Synthetic Minority Over Sampling Technique\\
\textbf{MLP} &  Multi-Layer Perceptron  \\
\textbf{GB} &  Gradient Boosting  \\
\textbf{SIFT} &  Scale‑invariant feature transform algorithm  \\
\textbf{SURF} &  Speeded up robust features algorithm  \\
\textbf{KAZE} &  Algorithm for features detection and description in nonlinear scale-space  \\
\textbf{ORB} & Algorithm to detect the key points in the image  \\
\textbf{GloVe} &  Word embedding techniques  \\
\textbf{Raspberry PI} &  IoT device which is a hardware platforms supported by Android Things  \\
\textbf{BiLSTM} &  Bidirectional Long Short‐Term Memory  \\
\end{abbreviations}

\newpage
\begin{Publications} % update by sabbah

    \item pub 1
    \item  pub 2

\end{Publications}


%----------------------------------------------------------------------------------------
%	PHYSICAL CONSTANTS/OTHER DEFINITIONS
%----------------------------------------------------------------------------------------

%\begin{constants}{lr@{${}={}$}l} % The list of physical constants is a three column table

% The \SI{}{} command is provided by the siunitx package, see its documentation for instructions on how to use it

%Speed of Light & $c_{0}$ & \SI{2.99792458e8}{\meter\per\second} (exact)\\
%Constant Name & $Symbol$ & $Constant Value$ with units\\

%\end{constants}

%----------------------------------------------------------------------------------------
%	SYMBOLS
%----------------------------------------------------------------------------------------

%\begin{symbols}{lll} % Include a list of Symbols (a three column table)
%$\pi$ & survival rate & \% \\
%$l$ & retirement choice & \\

%$a$ & distance & \si{\meter} \\
%$P$ & power & \si{\watt} (\si{\joule\per\second}) \\
%Symbol & Name & Unit \\

%\addlinespace % Gap to separate the Roman symbols from the Greek

%$\omega$ & angular frequency & \si{\radian} \\

%\end{symbols}

%----------------------------------------------------------------------------------------
%	DEDICATION
%----------------------------------------------------------------------------------------

%\dedicatory{For/Dedicated to/To my\ldots} 

%\fi %uncomment together with \Leftrightarrowalse above to displaye

%----------------------------------------------------------------------------------------
%	THESIS CONTENT - CHAPTERS
%----------------------------------------------------------------------------------------

\mainmatter % Begin numeric (1,2,3...) page numbering

\pagestyle{thesis} % Return the page headers back to the "thesis" style

% Include the chapters of the thesis as separate files from the Chapters folder
% Uncomment the lines as you write the chapters

% Chapter Template
% !Tex root = main.tex

% Chapter 1

\chapter{Introduction} % Main chapter title
\Quote{“The knowledge of anything, since all things have causes, is not acquired or complete unless it is known by its causes”.
\\
-Avicenna (Ibn Sina)}


The primary objective of this dissertation is to advance the field of science by developing innovative methods for extracting relationships from Arabic text. In Section \ref{}, we present the scope and motivation behind this research. Section \ref{} summarizes the research problem and questions, while Section \ref{} highlights our contributions and progress. Finally, the overall structure of the dissertation is presented in Section \ref{}.


\label{Chapter1} % For referencing the chapter elsewhere, use \ref{Chapter1} 

%----------------------------------------------------------------------------------------

% Define some commands to keep the formatting separated from the content 
\newcommand{\keyword}[1]{\textbf{#1}}
\newcommand{\tabhead}[1]{\textbf{#1}}
\newcommand{\code}[1]{\texttt{#1}}
\newcommand{\file}[1]{\texttt{\bfseries#1}}
\newcommand{\option}[1]{\texttt{\itshape#1}}

%------------------------------------------------------------------------



\section{Motivation and Scope}
\label{sec:scope_motivation}


Relation Extraction is a fundamental task in Natural Language Processing (NLP) aimed at identifying and classifying the relationships between entities in a text. This process involves extracting pairs of entities along with the relationships that connect them, typically represented as triplets $\langle \text{subject}, \text{relation}, \text{object} \rangle$ \cite{}. These triplets capture the semantic connections and interactions between the entities~\cite{}. Relation extraction is essential in a variety of applications, such as knowledge graph construction, where it helps create structured and queryable information representations ~\cite{}. It also plays a critical role in event detection by identifying event arguments and roles to describe the events in a text \cite{}. Furthermore, relation extraction enhances the performance of question-answering systems by linking entities in queries to relevant answer entities, thus improving the accuracy of fact-based responses~\cite{}.

Relation extraction can be performed at both the sentence level and the document level \cite{}. In sentence-level relation extraction, the target entities are within the same sentence (Figure \ref{fig:sentence_level}). In contrast, document-level relation extraction (Figure \ref{fig:document_level}) involves entities that span multiple sentences, making the extraction process more challenging. This is due to the need to reason over longer contexts and resolve entity co-references. 

% write a para about the state of the art in general and for Arabic. 

% The gaps in general 

% Scope:
% within sentence level or document level?  = sentence
% MSA and dialects? = MSA
% named entities and relations together or not?  = joint
% which type of relations? and how many relations? and how many entity types? = fine-grain with almost all common relations.




\begin{figure}
  \centering
  \includegraphics[width=1\linewidth]{Images/sentence_level_smilar.pdf}
  \caption[Sentence-level relation extraction]{Sentence-level relation extraction example, with three entities: "Rebecca Storm", "West Yorkshire", and "England", and two distinct relations: "Has-Birthplace" and "Located-in".}
  \label{fig:sentence_level}
\end{figure}

\begin{figure}
  \centering
  \includegraphics[width=1\linewidth]{Images/document_level_1.pdf}
  \caption[Document-Level Relation Extraction]{Document-level relation extraction example, with three entities: Ahmed, Amazon, and Seattle, Washington. Ahmed is mentioned twice, denoted as "Ahmed" and "He", indicating coreference.}
  \label{fig:document_level}
\end{figure}


Deep learning has become the dominant paradigm in relation extraction, particularly for processing large-scale corpora. Early methods focused on training models such as Recurrent Neural Networks (RNNs) \cite{} and Long Short-Term Memory (LSTM) networks \cite{}, or fine-tuning small language models like Bidirectional Encoder Representations from Transformers (BERT) on specific domain datasets \cite{}. While small language models demonstrated promising performance at the sentence level, their effectiveness diminished at the document level or when applied to out-of-domain datasets \cite{}. This limitation is largely due to the restricted knowledge embedded in small language models during pretraining, which hinders their ability to generalize beyond the training data. Chapter \ref{} will provide a detailed analysis of this issue and its impact on relation extraction performance.

In contrast, Graph Neural Networks (GNNs) have emerged as a powerful alternative, which demonstrates superior performance at the document level \cite{}. By incorporating graph structures, such as dependency and syntactic trees, GNNs capture complex relationships between words and entities within a text, providing richer contextual representations. This capability is especially valuable in relation extraction tasks, where the intricate relationships between entities require effective modeling of pairwise connections.

The introduction of Large Language Models (LLMs), such as OpenAI's GPT-4 \cite{} and Meta's LLaMA 3.3 \cite{}, has further advanced the relation extraction paradigm by shifting the focus from fine-tuning to in-context learning. LLMs exploit the extensive knowledge encoded during pre-training on large-scale, diverse datasets, which span a wide array of domains and tasks \cite{}. This enables LLMs to effectively tackle challenges such as the scarcity of annotated datasets, and poor performance on out-of-domain data.

Recent advancements have demonstrated the potential of LLMs in zero-shot and few-shot relation extraction settings, where they identify relationships between entities in a text without requiring extensive data annotation \cite{}. Current methods often reformulate the relation extraction task as a Question-answering problem \cite{}, converting sentences into queries and candidate relations into options. Techniques such as self-consistency within question-answering frameworks have been applied to improve prediction reliability by reducing uncertainty through majority voting. Strategies like task-aware demonstration retrieval, gold label-induced reasoning logic, and self-prompting have also been developed to enhance LLM performance by tailoring synthetic relation extraction data to specific relations \cite{}.

Despite the advancements in using LLMs for relation extraction, several challenges remain unaddressed. In-context learning for relation extraction is often suboptimal due to inadequate prompting strategies that fail to effectively capture the complexity of relations and the reasoning required within query sentences \cite{}. 
Moreover, Arabic relation extraction has received little attention in the literature, with limited exploration of LLMs for this purpose. The scarcity of high-quality, annotated Arabic datasets and the absence of effective pre-trained models tailored for Arabic further compound these challenges, leaving a significant gap in research on relation extraction for underrepresented languages \cite{}.

This research aims to develop an effective approach for extracting relationships at the sentence level in Modern Standard Arabic (MSA) using LLMs. Given the complexity of Arabic, including its rich morphology and syntactic variations, accurately identifying relationships within sentences remains a significant challenge \cite{}. However, existing studies often overlook these linguistic complexities or rely on coarse-grained relation types, limiting their applicability in real-world scenarios. To address this gap, this research focuses on extracting the most commonly used fine-grained relationship types, such as family relations, entity affiliation, and location relations, which are essential for various NLP applications \cite{}. The findings will contribute to advancing Arabic NLP, with a particular emphasis on knowledge graph construction.


\vspace{.9cm}

\noindent\textbf{\large Research Objectives:}
\vspace{.5cm}


\noindent This research proposes to address the Relation Extraction task using in-context learning in few-shot settings and aims to enhance existing few-shot learning approaches. The proposed approach will incorporate adaptive feedback mechanisms and error-aware retrieval strategies to improve the performance of LLMs in handling Relation Extraction. Specifically, the objectives of this thesis are:

\begin{enumerate}
    \item \textbf{Joint Entity and Relation Extraction:} Propose a shift from entity-centric representations, which are primarily designed for relation classification, towards relation-aware embeddings. These embeddings would facilitate joint entity and relation extraction while enabling the capture of multiple relations in a single instance. This proposed approach aims to improve the model’s flexibility and scalability in handling complex, real-world Relation Extraction tasks.
    
\item \textbf{Optimized Demonstration Retrieval}: Improving the selection of informative examples to enhance few-shot learning capabilities, ensuring that the model is guided by diverse and contextually relevant relational patterns.

\item \textbf{Enhanced Model Robustness}: Strengthening the model’s ability to handle ambiguous and uncertain relational patterns by incorporating more comprehensive and challenging training examples.

\item \textbf{Improved Reasoning Capabilities}: Refining the model’s decision-making process through adaptive feedback mechanisms, enabling more accurate and context-aware relation extraction.
\end{enumerate}

\section{Problem Statement and Research Questions}\label{sec:research_problem}

Relation extraction aims to identify semantic relationships between entities mentioned within a text, and classify these relationships into a pre-defined set of relation types \cite{}. Named entities refer to specific real-world objects, such as persons, organizations, locations, dates, and other proper nouns that carry distinct semantic meaning \cite{}. Formally, relation extraction can be formally defined as a function \( F: S \rightarrow R \), where \( S \) represents the text (i.e., sentence(s)), and \( R \) represents the set of target relations of interest (i.e., those we aim to extract). Thus, a relation \( r \in R \) is a triplet of the form \( \langle e_{i}, r, e_{j} \rangle \), where \( e_{i} \) and \( e_{j} \) are the named entities mentioned in the text \( S \).\\
A key challenge in relation extraction is the need for an integrated approach that simultaneously identifies entity spans and their semantic relationships \cite{}. Existing methods often rely on predefined entity types, limiting their adaptability to real-world applications. This research focuses on joint entity and relation extraction, where the goal is to directly identify entity spans within the text rather than classifying entities into predefined categories \cite{}. This approach enables a more flexible and context-aware extraction process.
\begin{comment}
This research proposes to address the Relation Extraction task using in-context learning in few-shot settings and aims to enhance existing few-shot learning approaches. The proposed approach will incorporate adaptive feedback mechanisms and error-aware retrieval strategies to improve the performance of LLMs in handling Relation Extraction. Specifically, the objectives of this thesis are:


\begin{enumerate}
    \item \textbf{Joint Entity and Relation Extraction:} Propose a shift from entity-centric representations, which are primarily designed for relation classification, towards relation-aware embeddings. These embeddings would facilitate joint entity and relation extraction while enabling the capture of multiple relations in a single instance. This proposed approach aims to improve the model’s flexibility and scalability in handling complex, real-world Relation Extraction tasks.

    \item \textbf{Enhancing Robustness in Uncertain Scenarios:} Propose the integration of balanced demonstrations, including ambiguous and uncertain cases, to strengthen the model’s ability to handle complex relational data. 

    
    \item \textbf{Incorporating Iterative Refinement and Reasoning:} Propose the integration of adaptive feedback mechanisms to enable iterative refinement of the model’s reasoning capabilities.
  

\end{enumerate}
\end{comment}

This research seeks to advance the field of Arabic relation extraction by addressing three areas of inquiry:

\vspace{0.9cm}
\noindent \textbf{{\large First: Constructing Arabic Relation Extraction Corpus }(Chapter\ref{}):}
\vspace{0.5cm}


\begin{description}
\item[RQ1:] \textbf{Why are existing Arabic relation extraction corpora insufficient for advancing relation extraction models?}

This question investigates the limitations of current Arabic relation extraction datasets, particularly their narrow scope, lack of diversity, and inability to capture the linguistic nuances of both Modern Standard Arabic and its dialects. It examines how these gaps hinder the performance and generalization of relation extraction models, especially in real-world applications.

\item[RQ2:] \textbf{What relation types should be included in a new Arabic relation extraction corpus to address these limitations?}

This question explores the specific relation types that need to be incorporated into a new, comprehensive corpus for Arabic relation extraction. It focuses on identifying key relations across diverse domains and linguistic features, ensuring the new corpus  captures a broader and more representative set of relations, thereby enhancing the accuracy, robustness, and applicability of relation extraction models in Arabic.

\end{description}
\vspace{0.9cm}

\noindent \textbf{{\large Second: Benchmarking LLMs on Arabic Relation Extraction }(Chapter\ref{}):}
\vspace{0.5cm}


\noindent We aim to address three key questions to evaluate the effectiveness and generalization capability of LLMs in Arabic relation extraction.

\begin{description}
    \item[RQ3:] \textbf{How effective are LLM-based models in Arabic relation extraction?}  

    
    This question arises from observations that LLM performance in English tasks remains unsatisfactory compared to models built on small language models \cite{}. The question arises from observations that LLMs often underperform on English tasks compared to models based on smaller language models \cite{}. As will be discussed in Section \ref{}, understanding the effectiveness of LLMs in Arabic is critical for assessing their broader applicability across languages and domains.


    \item[RQ4:] \textbf{How does the number of relation types affect the performance of LLM-based models?}

    
    Most benchmark datasets for relation extraction contain a limited number of relation types, overlooking real-world scenarios where the number of relations is much larger \cite{}. Investigating how LLMs perform with a large number of relation types is crucial for assessing their scalability.

    \item[RQ5:] \textbf{How does instruction-tuning affect the generalization capability of LLMs compared to in-context learning? }

    
    Updating model weights through instruction-tuning has the potential to enhance task-specific performance \cite{}. However, it also introduces the risk of catastrophic forgetting, where performance on previously learned tasks deteriorates. As will be explored in Sections \ref{} and 
 \ref{}, this presents a significant trade-off: while instruction fine-tuning may improve generalization within a specific domain, it could compromise the model's broader adaptability. Understanding this trade-off is essential to evaluate the practicality of instruction-tuning compared to in-context learning.
\end{description}


\vspace{0.9cm}

\noindent \textbf{{\large Third: Proposing LLM-based joint entity and relation extraction }(Chapter\ref{}):}
\vspace{0.5cm}

This research aims to explore new methods for improving relation extraction in few-shot learning settings. The following research questions guide this research, which addresses key challenges in enhancing model performance:



\begin{description}
    \item[RQ6:] \textbf{How can LLM prompts be effectively tuned for datasets with numerous relations?}  
    
    This question explores how prompt engineering techniques can be adapted to handle datasets containing a large and varied set of relations. The challenge is to design prompts that effectively capture the nuanced relationships between entities across different contexts, ensuring that the model performs well across a broad spectrum of relation types. This question investigates how prompt structure, wording, and the inclusion of examples can influence model performance in relation extraction tasks involving multiple relations. Additionally, it examines how these techniques can address issues like ambiguity in relations and ensure consistent model predictions in real-world scenarios.



    \item[RQ7:] \textbf{How do ambiguous and uncertain cases in training data impact a model’s ability to extract relationships?}

    
    This question explores how the presence of ambiguous and uncertain examples in training data influences a model's performance in extracting relationships. It investigates whether exposing models to noisy, incomplete, or conflicting instances helps improve their robustness, making them more capable of dealing with real-world challenges where data may be unstructured or ambiguous.

    \item[RQ8:] \textbf{How does feedback improve LLM performance in relation extraction?}
    
    This research question explores the impact of iterative feedback mechanisms on the LLMS in relation extraction tasks. Specifically, it investigates how the incorporation of adaptive feedback during the extraction process can refine the model’s predictions, enhance its ability to manage complex and ambiguous relationships, and improve its overall generalization to diverse real-world scenarios. Additionally, this question seeks to understand the effect of multi-turn feedback loops, and direct preference optimization on the model’s accuracy, robustness, and scalability across various relation types, particularly in low-resource languages like Arabic \cite{}.


\end{description}


\section{Contribution}
\label{sec:contribution}
This dissertation addresses the challenges of relation extraction in Arabic through three key contributions: (i) the construction of a high-quality, manually annotated Arabic relation extraction corpus, (ii) the novel benchmarking of LLMS on Arabic relation extraction, and (iii) a new LLM-based framework to enhance performance in Arabic Relation extract. Each contribution will be discussed in detail in the following chapters.

\vspace{.9cm}
\noindent \textbf{\large 1. Construction of Arabic Relation Extraction Corpus ( Chapter\ref{})} 
\vspace{.5cm}


This contribution addresses \textbf{RQ1} and \textbf{RQ2} by introducing Wojood\textsuperscript{Relations}, the first manually curated Arabic relation extraction dataset, annotated by native speakers to overcome these challenges. Wojood\textsuperscript{Relations} is the largest and most comprehensive Arabic relation extraction corpus, comprising 33K sentences (550K tokens) with 40 relation types across diverse domains, covering both Modern Standard Arabic (MSA) and dialects. Our analysis demonstrates that Wojood\textsuperscript{Relations} exceeds existing datasets in both scope and quality, providing a solid foundation for benchmarking and advancing relation extraction models. \\

\textit{The corpus is 90\% completed and I am writing an article about it}

\vspace{.9cm}
\noindent{\large \textbf{2. Benchmarking LLMs on Arabic Relation Extraction (\S \ref{}) }}
\vspace{.5cm}


This contribution addresses \textbf{RQ3}, \textbf{RQ4}, and \textbf{RQ5} by systematically evaluating the performance of large language models (LLMs) on Arabic relation extraction tasks. By comparing zero-shot, few-shot, and fine-tuned settings, as well as exploring advanced prompting strategies like CoT and RAG, the study provides insights into the trade-offs between generalization and task-specific optimization. Additionally, it investigates how ambiguous, uncertain, and diverse data affect the robustness of LLMs in relation extraction, aiming to determine the most effective model configurations for this specialized task. 



\textit{The benchmarking is 50\% completed and I am writing an article about it}




 %Our analysis underscores the unique challenges of Arabic relation extraction, including the influence of handling numerous relation types on model performance.

\vspace{.9cm}
\noindent{\large \textbf{3. LLM-Based Framework for Arabic Relation Extraction (Chapter \ref{})}}
\vspace{.5cm}

This contribution addresses \textbf{RQ6}, \textbf{RQ7}, and \textbf{RQ8} by proposing a multi-component model for few-shot entity and relation extraction in Arabic, leveraging LLMs to enhance performance. The model investigates how LLM prompts can be effectively tuned to handle datasets with numerous relations, exploring prompt engineering techniques to capture the nuances of diverse relation types across contexts (RQ6). It also examines the impact of ambiguous and uncertain examples on the model's ability to extract relationships (RQ7), and evaluates the role of iterative feedback and multi-turn prompts in improving extraction accuracy and model refinement (RQ8).
%Few-shot entity and relation extraction in Arabic presents unique challenges due to the complexity of the language. Existing methods often struggle to balance generalization with precision, particularly in low-resource scenarios. 

%In this dissertation, we propose a multi-component model for few-shot entity and relation extraction, designed to address these challenges when leveraging LLMs. The approach begins with a relation classification module that filters out confident and uncertain relations in the test sentence. It then employs a demonstration retrieval module to select the most relevant examples, which are then organized into support sets and query sets. Finally, a teacher-student framework is proposed using a multi-turn prompt. In the first turn, the teacher module leverages the support set for guidance and the query set for an initial assessment, generating feedback.  The student then uses the feedback to construct the second-turn prompt, helping the model refine its predictions on the test sentence. This pipeline enhances relation extraction by effectively utilizing few-shot demonstrations and feedback mechanisms.


By addressing these research questions, the dissertation aims to provide a comprehensive evaluation of current LLM capabilities in Arabic Relation Extraction while proposing and validating innovative solutions to improve their performance.


\section{Dissertation Structure}
\label{sec:thesis_structure}

This dissertation is structured into five chapters. Chapter \ref{} provides background information and a review of related literature. Chapter \ref{} discusses the construction of the Relation Extraction corpus used in this research. Chapter \ref{} focuses on the benchmarking of LLMs in the context of Arabic relation extraction, evaluating their performance and limitations. Chapter \ref{} introduces the proposed methodology for improving relation extraction in few-shot settings.


% Chapter Template
% !Tex root = main.tex

% Chapter 2

\chapter{Background and Related Work}
\label{Chapter2}
\Quote{In this section you need to explain all the theory required to understand your dissertation (i.e.\ the following chapters). In this section you need to explain all the theory required to understand your dissertation (i.e.\ the following chapters). In this section you need to explain all the theory required to understand your dissertation (i.e.\ the following chapters). In this section you need to explain all the theory required to understand your dissertation (i.e.\ the following chapters).}



This chapter presents an extended analysis of the goals of this thesis and the motivation driving this endeavor. We investigate and specify several
challenges in ontology engineering. Section 2.1 discusses the significance,
and challenges of ontology reusability. In section 2.2, we introduce and
discuss the most challenging issue in ontology engineering: the
application-independence of ontologies. In section 2.3, we clarify some
ontology evolution challenges. To end, section 2.4 draws some
conclusions and derives the main ontology engineering requirements. 

\section{Background}
\subsection{The Android Operating System}
Android is an open-source operating system proposed by Google that is based on a modified version of Linux. Android applications are built with Java, Kotlin, or other cross-platform frameworks. Finally, the applications are compiled in Java, but the original libraries are written in C/C++ \cite{AcharyaRB22}. 

The Android architecture is divided into four layers, as shown in the figure \ref{SampleData}. The first layer is the Android user applications that include the native Android application and the third-party application. In this layer, the user can install and interact with applications. 

\begin{figure}[H]
\centerline{\includegraphics[scale=0.40]{Images/AndroidArch.png}}
\caption{Android architecture}
\label{SampleData}
\end{figure} 

The application framework contains a set of components that can be reused to provide a service using an API. The \textbf{Notification manager} is a service used by all Android applications to display the custom message display in the status bar. The \textbf{Resource manager} provides access to non-code resources such as color settings, layout files, strings, and user interface layouts~\cite{Platform37:online}. The \textbf{Content providers} are used to share data between applications. The \textbf{ Activity manager} is used to manage all activities of an application and to record the activities in a log file to help debug the app if it crashes~\cite{AcharyaRB22}. The \textbf{View system} service is an extensible set of views that can be used to build an application user interface. The \textbf{Location manager} is used to give access to the location of the device. The \textbf{window manager} is a service responsible for controlling the ordered list of windows, which windows are visible, and how they appear on the screen. Finally, the \textbf{package manager} provides version information on the Java package implementation and specification. The package manager is stored in the manifest file, which is distributed with the classes. 

The libraries layer includes Java libraries that are specific to Android development. The following are some of the core Android libraries:



\begin{itemize}
    \item \textbf{App:} It is the foundation of all Android apps and gives access to the application model. 
    
    \item \textbf{SSL:} Secure Sockets Layer functionality that leverages a combination of Java Cryptography Architecture (JCA) and the Android framework's TLS (Transport Layer Security) implementation for secure communication protocols.

    \item \textbf{Content:} Enable access to content data between applications and app components.
       
     \item \textbf{Media Framework:} provides functionalities for working with multimedia content like audio, video, and images.
       
    \item \textbf{Webkit:} Provide a set of classes to make it possible for applications to include web browsing functionality.
       
     \item \textbf{Security:} This library collection within AndroidX offers various security-related functionalities like secure networking and data encryption.
\end{itemize}

The last layer is the Linux Kernel which is considered a core of the Android architecture. It interacts with device drivers and manages all the available drivers such as camera drivers, wireless drivers, audio drivers, memory drivers, etc. These drivers are required during runtime. The Linux Kernel provides an abstraction layer between the device hardware and the other components of Android architecture.

\subsection{Android Package Kit (APK)}
The Android Package Kit (APK) is an archive file format for applications that work on the Android operating system. The APK file can be installed from the official Android market or from a third-party site. If an Android user wishes to install APK files from a source other than the official app store, they can grant permission on their device to install apps from unknown sources. 
The main components of the APK file are as follows:
\begin{itemize}
    \item \textbf{AndroidManifest.xml:} It contains XML metadata information about components of the application, such as security permissions, activities, and services.
    
    \item \textbf{Classes.dex:} This file contains the source code of an application written in Java and compiled to a Dalvik executable with \textit{.edx} extension.
    
    \item \textbf{Resources.arsc:} A binary XML file that includes precompiled application resources~\cite{Ali19}.
    
    \item \textbf{Resources (res/):} A folder containing none-compiled resources that the application requires at run time, such as menus, images, layouts, and database use.
    
    \item \textbf{Assets (assets/):} optional folder containing application assets that Asset Manager can retrieve.
    
    \item \textbf{Libraries (lib/):} optional folder containing code compiled that is particular to various processors, such as arm and x86 \cite{Ali19}.
    
    \item \textbf{META-INF:} folder containing the MANIFEST.MF file. Additionally, it includes the application developer's signature. The signature can be used to authenticate an external developer.
\end{itemize}

\subsection{Android Application Components}
The Android platform provides two types of components that can be used in Android applications: main components and additional components. The main components contain five different types of component~\cite{CompApplicat75:online, AcharyaRB22}

\begin{itemize}
    \item \textbf{Intent:} It is a messaging component to send messages between other components of the app ~\cite{AcharyaRB22}. Two types of intents: implicit and explicit. Implicit intent is used to interact with other applications, while explicit intent passes the message between the same application components.
        
    \item  \textbf{Activities:} 
    Activities are single screens with a user interface. The Android app starts at the main activity defined in AndroidManifest.xml, then the app can open additional activities. 
       
     \item \textbf{Services:} They deal with the background processing of an application. For example, a service might play music in the background. 
       
    \item  \textbf{Broadcast receivers:} Respond to broadcast messages coming from external applications or system applications. For example, the application can notify other applications that newly downloaded data are available for use.
       
      \item \textbf{Content providers:} Manages shared data that can be stored in the database, file system, on the Web, or any other constant storage location that the application can access. 
\end{itemize}


\subsection{Android Security}
Android being an open source platform, this makes it a target for hackers. Therefore, the security of applications is a significant concern and a challenge. Android provides two levels of security: at the Linux kernel level and at the application level~\cite{VishnoiMNP21}. The security mechanisms of Android include the following features:


\begin{itemize}
    \item \textbf{Sandbox:} Each application is assigned a unique user ID (UID) and runs in its own process and isolates space. Thus, one application with its own UID did not allow it to be accessed or modified by another application. Additionally, the application running in its own sandbox has limited access to system resources. 
    
    \item \textbf{Permissions:} Any application must define the permission needs of other application components or Android resources in the AndroidManifest.xml file. The user must grant these permissions during app installation; otherwise, the application cannot be installed on the device. Moreover, the permission can be granted when the application is run and requested in runtime, which is called dynamic permissions.
    
     \item \textbf{Signatures:} The Android application must be stamped and digitally signed with a certificate that identifies the developer of the application. A developer with the same certificate can update the application in the future. Also, the applications with this signature can trust each other by sharing UID between them.
     
     \item \textbf{Secure Inter-process Communication:} Secure Inter-process Communication (IPC) protocols allow an application to communicate with remote servers and also other applications.
\end{itemize}

\subsection{Mobile Malware Overview}
The term malware is derived from \textbf{mal}icious soft\textbf{ware}. Mobile malware is "gaining access to a device for the purpose of stealing data, damaging the device, or annoying the user, etc" \cite{FeltFCHW11}. In the context of mobile, malware apps can be installed from official stores, third parties, or use social engineering strategies\cite{Alzubaidi21}, in order to gain unauthorized access and use root privileges without the user's permission \cite{FeltFCHW11}. Attackers scan to find vulnerabilities in the Android system to exploit them and infect mobile devices with malware.

The popular types of mobile malware in the literature~\cite{QamarKC19} include the following types, which we describe in more detail for completeness.

\BfPara{Viruses} A virus is a piece of software that can replicate itself and propagate to different programs on a device~\cite{QamarKC19}. Viruses attach themselves as a piece of executable code in the application, and they need human interaction to run. Viruses are capable of stealing data, damaging host systems and networks, building botnets, stealing money, and displaying ads~\cite{CommonMa64:online}. An example of an Android virus is a Gazon attack that sends an SMS link to the mobile and shows as an app that gives Amazon rewards and vouchers worth up to \$200 ~\cite{GazonAndroidS7:online}  
    
\BfPara{Rootkits} A type of malware that obtains remote access and control of a device to exploit users. A rootkit uses an obfuscation technique to hide its presence in order to remain in the system for a longer time. The HummingBird is a complex mobile malware application that tries to steal credentials and generate fake advertisements~\cite{1NewMess66:online}.

\BfPara{Worms} It is a standalone malicious software with replication capability spread over the network and from device to device without user intervention. The worm may contain \textit{"Payload"} that is used to describe what the worm is prepared to do on the victim's device, such as damage user data and theft of confidential information. 
    
\BfPara{Trojans} This type of malware appears as a benign application, and the user is attracted to downloading and installing the malware through this application~\cite{QamarKC19}. The attacker can gain remote access to the device and steal information and data and delete or modify files. GnatSpy~\cite{NewGnatS80:online} is an example of the mobile Trojan malware family.
    
\BfPara{Spyware} The aim of this malware is to monitor the activities of users without their permission. These activities include screen watching, key logging, and stealing information from credential accounts. Spyware attaches itself to benign applications to exploit vulnerabilities. Acallno and FlaxiSpy are types of mobile spyware~\cite{QamarKC19}.
    
\BfPara{Botnets} It is an abbreviation for Robot Network, which is a collection of malware-infected devices connected to the Internet. The Botnet builds a large controlled network of devices that can be used for a variety of cybercriminal activities. These devices can be accessed and controlled by a botmaster that is installed on the attacked server as part of the infected network. With the help of the botmaster, the attacker can install new malware on the victim device without the user's acceptance~\cite{HijawiAAHF21}.
    
\BfPara{Adware} This malware is designed to send advertisements to users spontaneously. Adware commonly shows ads as pop-up ads. Adware infiltrates the device by clicking ads. Then the adware can install malicious software to steal the user's information or track him.
    
\BfPara{Ransomware} This type of malware encrypts all data, locks the device, and displays messages to force users to pay. The attacker then sends the decryption key to victims to restore their own data. Lockdroid, Xbot, Simplocker, and adult player are examples of Android ransomware \footnote{https://themerkle.com/top-4-types-of-android-ransomware/}.
      
\BfPara{Backdoor} A type of malware that is designed to open a network door to allow other malware to enter the device. Recently, xHelper has become an ``unkillable'' Android backdoor that forces Android users to reset their mobiles to remove this malware \footnote{https://arstechnica.com/information-technology/2020/04/solved-how-android-backdoor-called-xhelper-survives-factory-resets/}.
       
\BfPara{Key-logger} The key-logger malware records everything the user types on the mobile. EventBot is a new type of Android banking malware that exploits accessibility features to steal sensitive data from financial applications, such as SMS two-factor authentication codes\footnote{https://thehackernews.com/2020/04/android-banking-keylogger.html}.


 \subsection{Android Platform Security Threats}
Attackers can exploit vulnerabilities in the Android system to infect mobile devices with malware. Bhat \etal provides a comprehensive analysis of various threats in the Android ecosystem. The study classifies threats into four types, including attacks on the hardware level, kernel-based, hardware abstraction layer-based, and applications-based level~\cite{BhatD19}. Android applications use the API to communicate between these layers, and attackers can penetrate Android at each level.

Karbab \etal presented a summary of threats to Android applications that include a limitation on mobile apps to check whether the applications deployed are malware or not. The official Android application market, such as Google's Play Store, could contain malware applications ~\cite{KarbabDDM21}. Furthermore, the Android app can be installed from external sources, such as SD cards, which some consider as an entry point for malware. Moreover, before installing applications, only 0.17 percent of Android users understand the permissions policies~\cite{KarbabDDM21}. Additionally, developers request permission to use the apps more than the apps need. It is relatively easy to repackage the APK file after reverse engineering a benign app and injecting it with malicious code ~\cite{KarbabDDM21}.

\subsection{Concept Drift}
The phenomenon of concept drift was first proposed in 1986 by Schlemmer \etal~\cite{SchlimmerG86}. It refers to the unexpected change in the statistical properties or defining features of the target variable over time in non-stationary data distributions. This presents a significant challenge for machine learning models that assume stationary input data distributions, where training and testing data are expected to be very similar~\cite{XiangZCW23}. In real-world scenarios, such as malware detection domain, the evolving nature of data can lead to concept drift, which impacts the accuracy and effectiveness of the model over time. Concept drift can arise in multiple situations, including changes in feature distributions. 

\BfPara{Concept Drift Root Causes} Three causes of concept drift proposed by Xiang \etal~\cite{XiangZCW23} based on the joint probability distribution included the following. 

\begin{itemize}
    \item Virtual concept drift: In situations where the probability of $x$ changes, while the probability of $y$ given $x$ remains unchanged, as shown in the equation~\ref{VirtualEquation}. In this case, the decision boundary remains unaffected, and only the feature space changes. In the malware context, this cause occurs when the malware evolves and the adversaries change the code of the app (static features) or behavior (dynamic features). But the malware still belongs to the same type and family.
    
    \begin{equation}
    \label{VirtualEquation}
P_{t0}(x) \neq P_{t1}(x) \quad \text{and} \quad P_{t0}(y|x) = P_{t1}(y|x)
\end{equation}


    \item Real concept drift: When the probability of
$y$ given $x$ changes while the probability of $x$ remains constant. This cause can be expressed using the equation~\ref{RealEquation}. This scenario directly affects the machine learning model, changing both the feature space and the decision boundary; for example, the emergence of a new malware family.

 \begin{equation}
    \label{RealEquation}
P_{t0}(y|x) \neq P_{t1}(y|x) \quad \text{and} \quad P_{t0}(x) = P_{t1}(x)
\end{equation}


    \item Hybrid concept drift: The last reason can lead to concept drift, which is both virtual and real concept drift and can exist in the data stream simultaneously.
     \begin{equation}
    \label{HybEquation}
P_{t0}(x) \neq P_{t1}(x), \quad P_{t0}(y|x) \neq P_{t1}(y|x)
\end{equation}

    
\end{itemize}


\BfPara{Concept Drift Types} Concept drift can take different shapes over time: abrupt, incremental, gradual, and recurring drift~\cite{XiangZCW23}. Each type represents a different shape of change in the fundamental concept of the data stream. Abrupt drift denotes sudden shifts from one concept to another in a short timeframe. However, incremental drift is similar but slow, and there are continuous shifts between concepts. Gradual drift presents periodic shifts between concepts. The last type is recurring drift, which includes the periodic reappearance of previous concepts over time~\cite{XiangZCW23}.

%https://www.mdpi.com/2076-3417/13/11/6515 

 

\chapter{Related Work}
\label{Chapter3}

Dealing with Android malware involves a multifaceted task, requiring consideration of various factors during the analysis and detection phases. Several approaches have been proposed to analyze Android malware. Most studies analyzed applications and classified them as benign or malicious. Malware analysis uses three main methodologies: static, dynamic, and hybrid approaches. Subsequently, these analysis techniques extract features from applications to identify and detect malware. In the static approach, the application is analyzed before it is installed on the device. The dynamic approach monitors the application's behavior at run-time, while the hybrid approaches combine both to identify malicious apps.

\section{Android Malware Analysis and Detection}
Android malware detection methods can be divided into three methods called ``network traffic based methods'' that monitor network traffic data and analyze these data~\cite{GargPS17,WangYCYZC18}. Wangc \etal proposed a method that analyzes the words related to sensitive information in the HTTP header of the traffic data. Malware transmits sensitive information via HTTP GET and POST. Thus, this stream of data that passes in the HTTP packets can be used to extract features that contain semantic text unique to mobile devices, such as ``IMEI,'' ``longitude,'' and ``latitude.'' This type of analysis can be considered dynamic analysis. However, this method cannot be used efficiently with new malware that encrypts payloads to keep their malicious code hidden \cite{WangYCYZC18}.
Garg \etal addressed the encryption data problem by comparing the network traffic patterns between benign applications and malware. But this method still ignores malware whose activities do not depend on network traffic~\cite{GargPS17}. Somarriba \etal proposed a dynamic analysis method for malware detection based on monitoring the behavior of applications at runtime. The proposed framework finds malicious URLs from the app traces on the mobile device and connects them to DNS service network traffic logs from the mobile operator. The limitation of this method is that it only considers URL and DNS traffic, while malware may use DNS tunneling techniques and HTT traffic~\cite{SomarribaZ17}. Moreover, the experiment was conducted on an emulator where the malware can be detected but not on a real device because the emulator does not support the anti-debugging evade method. Additionally, malware that uses certificate pinning can evade detection. The second and third methods are ``inner interaction based''~\cite{CaiMRY19} and ``permission-based''~\cite{AroraPC20}. These two methods focus on the Android system and the application components. In the next sections, we will present several analytical and detection approaches based on these methods at the Android mobile level. It is worth mentioning that there are unique characteristics for Android malware, which was investigated by Alasmary \etal~\cite{AlasmaryA0CNM18} who proposed a comparative study of the Internet of Things (IoT) and Android malware using graph-based analysis techniques. The authors leverage control flow graphs (CFGs) to represent the structural properties and behaviors of malware samples. They extract CFGs from IoT and Android malware samples and compute various graph metrics, to characterize and compare the malware families. The analysis indicated significant differences between IoT and Android malware in terms of their graph properties~\cite{AlasmaryA0CNM18}. 

\subsection{Static Approach}
The static analysis approach disassembles and decompiles code without running the application. Moreover, the static approach extracts the feature modalities from the APK using various methods, which are used for further applications such as malicious app classification and family attribution. Alzubaidi~\cite{Alzubaidi21} surveyed the literature in this space and highlighted three broad feature extraction methods: signature-based, permission-based, and Dalvik bytecode. In the study~\cite{KarbabDDM21}, it added a resource-based method and gave the name semantic-based instead of Dalvik bytecode. Vishnoi \etal used the term ``misuse detection,'' which is synonymous with (knowledge-based or signature-based). Additionally, ``anomaly detection'' is the same as the behavior-based approach in the dynamic approach ~\cite{VishnoiMNP21}. Regardless of which method is used, the main objective of static analysis is to extract features that will be used in detection models, such as machine learning or statistical methods for malware detection. 

% \citeauthor{WuZ021} in survey' \cite{WuZ021} paper summarized the app's components and static features as shown in Figure \ref{fig:compStatic}.  

% \begin{figure}[H]
% \centerline{\includegraphics[scale=0.50]{Images/compStatic.png}}
% \caption{ App's components and static features. \cite{WuZ021}}
% \label{fig:compStatic}
% \end{figure}


\BfPara{\ding{172} Signature-based method} Signature-based detection generates a unique signature pattern for known malware applications, the features extracted from the characteristics of Android applications such as permissions, broadcast receivers, content strings, or bytes \cite{LiFWCZYWG22}. The signature app compares to the malware signature library, which includes a unique signature for each known Android malware. \citeauthor{NgamwitrojL18} proposed a signature-based malware detection method that includes the permission and transmission data extracted from the manifest file. Malware signatures were created from 800 applications. The result of the detection of malicious signatures in the applications was 86. 56\% accuracy \cite{NgamwitrojL18}. \citeauthor{TchakounteNKU21} proposed a LimonDroid, an Android system that combines fuzzy hashing with YARA rules and VirusTotal signature schemes to improve the signature database used to identify Android malware applications. LimonDroid has been tested with 341 malicious and 300 benign applications, achieving an accuracy of 97.8\%. Comparison with similarity-based solutions shows that LimonDroid is more efficient in identifying Android malware\cite{TchakounteNKU21}. The study's main goal was not to suggest an improved detection method compared to existing ones but rather to create a strong signature database capable of accurately recognizing malicious trends in Android applications \cite{TchakounteNKU21}.


\BfPara{\ding{173} Permission-based method} Ilham \etal proposed a method for detecting Android malware in mobile applications. The permissions were extracted from the manifest file. Three feature selection algorithms were used: filter, wrapper, and embedded, to select 74 permission features. There were 731 malware and benign samples, with 673 malware and 58 benign. The best performance of the 10-fold cross-validation was 0.98 using all permissions with the Random Forest (RF) and SMO algorithms. Additionally, the effect of the feature selection algorithms on the result was negative, with an accuracy of 0.93. This study used a small number of samples and an unbalanced data set ~\cite{IlhamAA18}. Kato \etal proposed an Android malware detection scheme that used the Composition Ratio (CR) of permission pairs. New databases were constructed for the CR composite from 40 permissions, including 19K malware and 11500 benign samples collected from Androzoo, Drebin, and VirusShare. The RF classifier detected malware with an accuracy of 0.97 using the Drebin dataset and 0.84 with the mixed dataset, which consists of 6000 malware applications and 6000 benign. The limitation of this study was a leak in the detection app with a small number of permissions and a limitation in the selection of dangerous permission; perhaps more permissions should be included~\cite{KatoSS21}. The study~\cite{ArifRMAIF21} suggested a fuzzy analytical hierarchy process technique based on risk to evaluate the Android mobile application and a mobile malware detection system based on multi-criteria decision-making. This study used a static analysis to extract features, including permission-based characteristics. Risk analysis makes mobile users more aware of the risks of approving every permission request. Risk is divided into four categories: extremely high, high, medium, and low.
The study~\cite{MatRKAF22} used static analysis to analyze permission-based features. Permission features were used with the Bayesian probability model. Two algorithms were used for feature optimization: information gain and $\chi^2$ (Chi-square) . These algorithms were combined with the Naive Bayes classifier and the best accuracy for 10000 samples (5000 malware and 5000 benign) obtained from the AndroZoo and Drebin databases achieved 0.91 accuracy.
 
Sahin \etal investigated the reduction of the dimension of the vector generated for malware detection. The machine learning algorithms used included the Multilayer Perception Model (MLP), Naive Bayes NB, Linear Regression, k-nearest neighbors (KNN), C4.5, RF, and Sequential Minimal Optimization (SMO). Static analysis was used to extract Permission features from the 2000 apps, half of which were malware and the other half benign. In the feature selection phase, linear regression for feature selection was used to reduce the features from 102 to 27. The MLP algorithm outperformed all other classifiers and achieved an accuracy of 0.96 with a few features~\cite{SahinKAK23}.

Some studies have combined the permissions features with other features extracted from the APK source files.  
 In~\cite{UroojSMAR22}, ensemble learning was used to classify malicious Android applications.  Seven feature sets were included: Permissions, API Calls, Intents, App Components, Packages, Services, and Receivers. The number of features that belong to these feature sets is more than 56000 extracted from 100000 applications. The final dataset was collected from three databases: MalDroid, DefenseDroid, and GD, and included 18578 malware apps and 5716 benign. The result of the learning of the ensemble model that included AdaBoost, SVM, RF, NB, KNN, and decision tree algorithms was 0.96. 
 
\citeauthor{ShatnawiYY22} used the machine learning algorithms SVM, KNN, and NB and a publicly available dataset. Two features, the selection permissions and API calls, are used to detect malware. The dataset (CICInvesAndMal2019) illustrated Android malware's two sets of features: permissions and API calls. The total number of features acquired from the dataset was 8111, including 2089 permissions and 6022 API call features. The best classifier was SVM, which achieved a 0.94 accuracy rate using permission features and a 0.83 when using API calls~\cite{ShatnawiYY22}.

\BfPara{\ding{174} Resource-based method} This analysis method depends on the meta-data describing app components defined in the manifest file, such as required permissions, activity, service, intent, and more. Urooj \etal used reverse engineering to extract features from the source code of an Android app. Seven feature sets were used and included:\textit{ Permissions, API Calls, Intents, App Components, Packages, Services, Receivers}. The number of features that belong to these feature sets is more than 56,000 extracted from 100,000 applications. Trying to access some permissions may indicate malware behavior. For example, SMS, MICROPHONE, CONTACTS, STORAGE, and LOCATION are classified as dangerous permissions. However, VIBRATE and SET WALLPAPER are permissions without risk. The intent can be considered related to malware depending on the phrase that is used, such as (android.net), which is linked to the network manager, and intents embedding (com.android.vending), for billing transactions. API calls are tools that are controlled by the Android operating system. Camera, SMS, Bluetooth, GPS, network, and NFC are examples of API calls. These resources must be defined in the manifest file for use. When API calls request device resources or sensitive information, they are commonly malware. getDeviceId(), sendTextMessage(), setWifiEnabled(), and execHttpRequest() are all sensitive APIs. The set of broadcast features is an example of data transfer between components of two applications. Broadcast Receiver enables Android applications to respond to external events such as turning on the phone, receiving a text message, or making a phone call~\cite{UroojSMAR22}. This study~\cite{UroojSMAR22} and after the features extracted using \textit{Jadx-Gui} tool for decompile APK, and \textit{Androguard} to create vector mapping. Machine learning is used to classify malicious Android applications. The final dataset was collected from three database: MalDroid, DefenseDroid, and GD, and included 18,578 malware apps and 5716. The results and findings were 96. 24\% accuracy in malware classification. Millar \etal presented a multiview CNN-based neural network for the static analysis of Android malware. The selection of features depends on the opcodes, permissions, and proprietary Android APIs. The study evaluation for the zero-day scenario with the Drebin and AMD datasets showed that the model performed weighted average detection rates of 0.91 and 0.81, respectively. For general malware detection, the F1 scores of 0.9928 and 0.9963 again on the Drebin and AMD datasets, respectively~\cite{MillarMRM21}. Dhalaria \etal proposed a framework that combined three static features extracted from \textit{classes.dex} and \textit{AndroidManifest} files. The features included API calls, permissions, and intents. These combined features were used in the machine learning algorithms and K-Nearest Neighbor and achieved accuracy 95.9\%~\cite{DhalariaG20}. Applications were collected from apkmirror, apkpure, and virus share, and the dataset in this study~\cite{DhalariaG20} was labeled according to the Avira Antivirus tool with a total of 3547 application samples. 

\BfPara{\ding{175} Semantic-based method} 

is a branch of lexical analysis that uses data from different sources to extract semantic information. Dalvik bytecode embeds semantic information, such as methods, classes, and instructions, to generate data flow graphs that detect privacy leakage and misuse of telephony services. Another approach, by Bai \etal, proposed a scheme that uses network traffic and converts it into text, in which the N-gram is used for feature representation and the CNN model for malware classification. This approach uses NLP to extract features from APK components that are then used to form the semantic text~\cite{BaiLLQH21}. Alternatively, Zhang \etal proposed an approach for Android malware detection that depends on the method-level correlation of abstracted API calls from applications. However, the features are extracted from the code file. Each method code is replaced by API calls to generate a set of abstracted API call transactions. The association rules between abstracted API calls are used to calculate a confidence level, which indicates the semantics of the behavior for an app~\cite{ZhangLZP19}. Mc \etal used natural language processing and machine learning, the features extracted from the raw opcode sequences of the Android app~\cite{McLaughlinRKYMS17}. Karbab \etal proposed the MalDozer framework to detect Android malware and family members using raw API method calls extracted from DEX assembly sequences and fed to deep learning~\cite{KarbabDDM18} .

\BfPara{\ding{176} Image-based method} In addition to the methods mentioned above, which are used in malware analysis and detection, other approaches have been used. For example, the image-based approach is one of these directions that use static analysis to extract features and convert them to an image fed to machine learning and deep learning~\cite{XingJEJW22}. Unver \etal proposed the malware detection approach using an image-based technique. The features are extracted from the Manifest.xml, DEX, and Resource.ARSC files for each Android app and transformed into grayscale images. The study used the local feature selection algorithm and the global feature selection algorithm to optimize the training data that were fed into machine learning \cite{UnverB20}. The study~\cite{YadavMRVP22} suggested EfficientNet-B4, a CNN model, to detect Android malware using image-based malware representations of the Android DEX file. 
An image-based approach may be affected by code manipulation techniques and code obfuscation, a weakness of almost all static malware analysis detection techniques~\cite{UnverB20}. The summarizing and comparison of these studies are shown in tables \ref{tab1:Static}, \ref{tab2:Static}

%\begin{landscape}
    
%\begin{adjustbox}{angle=90}
%\begin{table}
\begin{sidewaystable}
\begin{center}
\caption{A comparison of a representative set of the literature work on static analysis approaches compared temporal across the utilized method, sample size, dataset name (wherever available), and the utilized features.}
%\renewcommand{\arraystretch}{1.5}
%\linespread{0.8}
\scalebox{0.75}{
\begin{tabular}{p{10em} p{2em} p{5em} p{10em} p{12em} p{25em}}
\hline
\textbf{Ref} &\textbf{Year} & \textbf{Method} & \textbf{Samples} & \textbf{Dataset} & \textbf{Features}  \\ \hline
Ngamwitroj \etal \cite{NgamwitrojL18} & 2018 & Signature  &  800 apps & NA &  Permission,
Broadcast-receiver \\
\hline

Tchakounte \etal \cite{TchakounteNKU21} & 2021& Signature & 341 M, 300 B & AMD (2019)  & App size, Permission, SHA1 and SHA256 from  app’s certificate.  Fuzzy hashing computed
from source file \\ 
 \hline
Ilham \etal\cite{IlhamAA18} & 2018  &  Permissions & 673 M, 58 B & AMD, Google Play & Manifest file, Permissions.  \\ 
 \hline
 Kato \etal \cite{KatoSS21} & 2021 & Permission & 19K M, 11500 B & Androzoo, VirusShare, Drebin & Permissions  \\
 \hline
Arif \etal \cite{ArifRMAIF21} & 2021 & Permission & 5K M, 5K B & Drebin, Androzoo & Permissions \\
\hline
Mat \etal \cite{MatRKAF22} & 2022 & Permission & 5K M, 5K B & Drebin, Androzoo & Permissions \\
\hline
Sahin \etal \cite{SahinKAK23} & 2021 & Permission & 1K M, 1K B & APKPure, AMD (2017) & Manifest file, Permission \\
\hline
Millar \etal \cite{MillarMRM21} & 2021 & Permission \& Resource & 41275 M & Drebin, AMD (2017), Malgenome,  Intel
Security dataset, Google Play &  opcodes, permissions, arbitrary API package, proprietary Android API package. \\
\hline
Urooj \etal \cite{UroojSMAR22} & 2022 & Permission \& Resource & 14078 M, 5716 B & MalDroid, DefenseDroid, Google Play & Permissions, API Calls, Intents, App Components, Packages, Services, Receivers. \\
\hline
Shatnawi \etal \cite{ShatnawiYY22} & 2022 & Permission \& Resource & 396 M, 1126 B & CICInvesAndMal2019 &  Permissions ,Intents, API calls. \\
\hline
 Dhalaria \etal \cite{DhalariaG20} & 2020 & Permission \& Resource & 3547 M\&B &  apkmirror, apkpure, virusshare & API calls, permissions, and intents\\ 
\hline
 Bai \etal \cite{BaiLLQH21} & 2021 & Semantic & 4354 M, 6500 B & CICAndMal2017 & Network traffic 19.2 GB for malware, and 19 GB for benign apps \\
\hline
 Zhang \etal \cite{ZhangLZP19} & 2019 & Semantic & 26031 M, 25833 B & AMD, Drebin, Androzoo  & API calls, Smali code, Methods name\\ 
 \hline
McLaughlin \etal \cite{McLaughlinRKYMS17} &2017 & Semantic & (1260 , 2475, 9902)  M, (863, 3627, 9268) B & Genome, Google play, Intel Security  & opcode sequences \\
\hline
Karbab \etal \cite{KarbabDDM18} & 2018 & Semantic & (33066, 20089) M, 37627 B,  & Malgenome, Drebin, MalDozer &  API method calls \\
\hline
Xing \etal \cite{XingJEJW22} & 2022 & Image & (8121, 8121, 5384) M, (7015, 2000, 5000) B & VirusShare, Google Play & Methods \\
\hline
Unver \etal \cite{UnverB20} & 2020 & Image & 4850 M, 4850 B & Drebin, AMD, Malgenom, Google play & Manifest, DEX, Resource files \\
\hline
\end{tabular}}
\label{tab1:Static}
\end{center}
\end{sidewaystable}
%\end{adjustbox}
%\end{landscape}


\begin{sidewaystable}
\begin{center}
\caption{A comparison of a representative set of the literature work on static analysis approaches compared Detection method, Result, Limitation.}
%\renewcommand{\arraystretch}{1.5}
%\linespread{0.8}
\scalebox{0.75}{
\begin{tabular}{p{10em} p{9em} p{8em} p{38em}}
\hline
\textbf{Ref} & \textbf{Detection Method} & \textbf{Results}  & \textbf{Limitation}  \\ \hline

Ngamwitroj \etal~\cite{NgamwitrojL18} & Statistical  & Accuracy: 0.865 & The signature constructed form only two features, permissions and broadcast-receiver. The data source of the applications was not known. Introduced only 17 malicious signature patterns \\
\hline

Tchakounte \etal~\cite{TchakounteNKU21}  & Rule based & Accuracy: 0.978 &  The algorithm used to measure the degree of similarity among applications was originally designed to identify shared characteristics in email spam messages.  \\ 
 \hline
Ilham \etal~\cite{IlhamAA18}  & ML &Accuracy: 0.98 & Only one feature type is used. The feature selection made the result worse than using all features  \\
 \hline
Kato \etal~\cite{KatoSS21} &ML & Accuracy: 0.973 & Leak of detection app with few permissions.Limitation in selecting dangerous permission; perhaps more permissions should be included.  \\
 \hline
Arif \etal~\cite{ArifRMAIF21} &Fuzzy AHP  & Accuracy: 0.9054 & Only focused on permission based features  \\
\hline
Mat \etal~\cite{MatRKAF22} &ML  &Accuracy: 0.91 & Only focused on permission-based features  \\
\hline
Sahin \etal~\cite{SahinKAK23}  & ML & Accuracy: 0.96 & Only focused on permission-based features \\
\hline
Millar \etal~\cite{MillarMRM21}  &ML  & Accuracy: 0.959 & Ignored protection levels and the expiration of permissions. \\
\hline
Urooj \etal~\cite{UroojSMAR22} & ML & Accuracy: 0.9624 & Realtime permissions requests and API requests were not included. \\
\hline
Shatnawi \etal~\cite{ShatnawiYY22} & ML &Accuracy: 0.94 & Not validate with up to date dataset   \\
\hline
Dhalaria \etal~\cite{DhalariaG20}  & ML  & Accuracy: 95.9 & Not explainable the low accuracy with intents features 0.345 while the combination reached 0.919  \\
\hline
Bai \etal~\cite{BaiLLQH21} &DL  &Accuracy : 0.926 &  Limitation when the network traffic uses encryption protocols  \\
\hline
Zhang \etal~\cite{ZhangLZP19} & ML & Accuracy: 0.98  & Use only API calls features.\\
 \hline
McLaughlin \etal~\cite{McLaughlinRKYMS17} &NLP, CNN  &Accuracy: 0.80-0.98 &  Low performance when testing different datasets, accuracy decay to 0.69 \\
\hline
Karbab \etal~\cite{KarbabDDM18} & DL &F1-Score of 0.96-0.99  &Not tested with external data  \\
\hline
Xing \etal~\cite{XingJEJW22}  & ML, DL & Accuracy: 0.96 & Complex model should be retrained to involve a new malware to detect it \\
\hline
Unver \etal~\cite{UnverB20}  & ML & Accuracy : 0.987 & Not tested with external data \\
\hline
\end{tabular}}
\label{tab2:Static}
\end{center}
\end{sidewaystable}


\subsection{Dynamic Approach} 
\hfill\\
This method analyzes an app's behavior in terms of its dynamic properties. Dynamic analysis can be used on the hardware level by collecting memory, CPU, battery, sensors, camera, and screen usage data. At the software level, features are extracted from network traffic, app patterns, information flows, privileges, permissions, data access, API calls, and different predefined functions~\cite{Alzubaidi21}.
Sihag \etal proposed a solution based on a dynamic analysis method to explore the behavior of the application~\cite{SihagSV020}. The Android logs at the kernel level are used to generate app signatures using Logcat. If the app attempts to leak information, jailbreak, gain access to dangerous permissions, or gain root privileges, the app is malware. Bhatia \etal used statistical analysis methods to understand the behavior of benign and malicious apps. The dataset was prepared from system call traces for both benign and malicious applications. Classic machine learning algorithms are used to classify apps according to their behavior~\cite{BhatiaK17}. Feng \etal propose EnDroid which is a dynamic analysis framework, with features extracted at runtime from system behavior traces and application-level malicious behaviors. EnDroid used the chi-square algorithm to identify dangerous dynamic behavior features~\cite{FengMSXM18}. 


The dynamic approach involves running the malware in a controlled environment to observe the app's behavior, whether on the host or network-based level~\cite{MohaisenAM15}. Several terms were introduced by the litterateur, such as behavioral patterns, taint analysis, anomalies, dynamic permission, etc. These terms are considered dynamic only if the data is collected in run-time. For example, dynamic permission can be known from the app files without running it; in this case, this method did not consider dynamic.  

Hu \etal study used system calls and network traffic to detect malware. The malware app dataset was collected from Contagio Mobile. A special algorithm was developed to detect the similarity between malware applications. Three malware categories were used to evaluate the proposed ShadowDroid tool: spyware, botware, and ransomware. ShadowDroid achieved an accuracy of 0.90~\cite{HuJC20}. Zhang \etal proposed framework involved two parts: the server, which has a model deployed to analyze system calls, and the client app, which was developed and installed on the Android device. The client app collects system calls and sends them to the server for analysis. Strace, Monkey Runner, and Android Studio emulator were used to design the framework. Then the system calls with natural language processing (NLP) techniques, TF-IDF and n-gram, for feature extraction. The apps were collected from the AMD dataset, and the framework achieved an accuracy of 0.99~\cite{ZhangMZRNJY22}. In~\cite{MahindruS17}, 123 dynamic permissions were extracted from 11000 Android apps from different available datasets. These features were used to evaluate five machine learning techniques. Feng \etal proposed EnDroid framework for dynamic malware detection, features were extracted from AndroZoo and Drebin datasets. DroidBox, Strace, and MonkeyRunner tools extracted system calls and other features. Then, nine machine learning algorithms were used to evaluate EndDroid, which achieved a rate accuracy of 0.983~\cite{FengMSXM18}. Guerra-Manzanare \etal introduced a comparative study that collected system call data from a real device and an emulated environment. The features were extracted from the system calls of both environments using ADB, Monkey-runner, Strace, Genymotion, and Android Studio. The study found significant differences in system call usage between real devices and emulators \cite{Guerra-ManzanaresV22} . In~\cite{BhatiaK17}, system calls were used to extract features from 50 malware and 50 benign applications during the application's runtime. Each application's frequency of system calls is considered the main set of features. Machine learning achieved an accuracy of 0.88 for classifying malware and benign applications. Casolare \etal proposed a method for Android malware detection. Dynamic analysis was used to extract the system call trace using ADB and Monkey-runner tools. These features of the system calls were used to create an image for each Android application. Therefore, the image generated from each application was used as input for the machine learning and deep learning classifiers. The detection model achieved 0.89. Most of the studies, as mentioned, used system calls as a data source to dynamically analyze the behavior of the app ~\cite{CasolareDIMMS21}.  

Some of the studies adopted other sources to extract the dynamic features, whether used alone or merged with other features. A framework proposed by \cite{MahdavifarKFAG20} for classifying malware into five categories: banking, riskware, SMS, adware, and benign. In addition, a new dataset was generated for dynamic and static features named CICMalDroid 2020, which includes 17,341 sample apps. The dynamic features included system calls, binder calls, inter-process communication (IPC), and remote procedure calls (RPC). This dataset was used to evaluate the framework using deep learning and machine learning. The best accuracy achieved was 0.97. In~\cite{WitBH22}, machine learning was evaluated using external datasets created by another study. The dataset was collected by real users in 2016. The features were extracted based on the hardware level, including battery, CPU, network, memory, I/O interrupts, and storage. These features were used to detect Trojan malware and achieved an accuracy of 0.72~\cite{WitBH22}.
 
Some studies have used static analysis to simulate dynamic behavior. Zhang~\etal used the static features of the API calls to simulate the behavior of malware apps. API calls and abstraction techniques extracted the features from the source code. Machine learning features were then used to detect malware~\cite{ZhangLZP19}. The same approach used by Onwuzurike \etal introduced the MaMaDroid system to detect Android malware. The system generates a Markov chain model based on API call sequences abstracted to a class, package, or family. The features were extracted using static analysis to simulate the behavior of malware. The output model is used well as input to machine learning classifiers~\cite{ OnwuzurikeMACRS19}. Furthermore, Shen \etal introduced a new technique for Android malware detection by analyzing complex data flows within applications. The core idea is to examine the structure and patterns of information flows within Android applications to identify malicious behavior based on the API call. The study used n-gram analysis to identify unique and common behavioral patterns present in complex flows. This n-gram analysis is performed on sequences of API calls that occur along the control flow paths of the complex flows. The authors evaluated their approach on 8,598 apps, consisting of recent and older generation benign and malware apps. They demonstrate the effectiveness of their technique in detecting malware in different generations of Android applications~\cite{ShenVMKZ19}.
The summarizing and comparison of most of these studies are shown in tables \ref{tab:dynamicTable}, \ref{tab:dynamicTable2}

\begin{sidewaystable}
\begin{center}
\caption{A comparison of a representative set of the literature work on Dynamic analysis approaches compared temporal across the utilized method, sample size, dataset name (wherever available), and the utilized features.}

\scalebox{0.75}{
\begin{tabular}{p{10em} p{2em} p{5em} p{10em} p{12em} p{25em}}
\hline
\textbf{Ref} &\textbf{Year} & \textbf{Method} & \textbf{Samples} & \textbf{Dataset} & \textbf{Features}  \\
\hline

Sihag \etal \cite{SihagSV020} & 2020 & Dynamic &  42 M, 260 B & NA M, Play store & log dumps fromlogcat \\
\hline

 Bhatia \etal \cite{BhatiaK17} & 2017& Dynamic & 50 M, 50 B & Genome, Play store & System calls \\ 
 \hline
Feng \etal\cite{FengMSXM18} & 2018  &  Dynamic  & (5213,5000) M,(5000,5000) B & AndroZoo, Drebin & System call.  \\ 
 \hline
 
Hu \etal \cite{HuJC20} & 2020 & Dynamic &130 M, 62 B & Contagio & Network packet, system call  \\
 \hline 
 Zhang \etal \cite{ZhangMZRNJY22} & 2022 & Dynamic &125  M,  300 B &AMD & System call \\
\hline
Mahindru \etal \cite{MahindruS17} & 2017 & dynmaic & 6971 M, 6029 B &appchina, hiapk, mumayi, gfan, pandaapp, slideme, Android Botne, Genome, DroidKin, AndroMalShare & Dynamic Permissions \\
\hline
Guerra \etal \cite{Guerra-ManzanaresV22} & 2022 & Dynamic & 8 M, 8 B &AndroidMalware 2020, CICAndMal2017 &  system calls \\
\hline
Casolare \etal \cite{CasolareDIMMS21} & 2021 & Dynamic &3355 M 3462 B & NA & System call. \\
\hline

 Mahdavifar \etal \cite{MahdavifarKFAG20} & 2020 & Dynamic & 17341 M, NA B & Proposed CICMalDroid2020 & System call, binder, IPC and RPC. \\
\hline
 Wit \etal \cite{WitBH22} & 2022 & Dynamic & NA & SherLock & Battery, CPU, network, memory, I/O interrupts and storage. \\ 
\hline
 Zhang \etal \cite{ZhangLZP19} & 2019 & Dynamic & (5.9K,5.6K) M, (20.5K,20.8K) B & AMD, Drebin & API calls \\
\hline
 Onwuzurike \etal \cite{OnwuzurikeMACRS19} & 2019 & Dynamic & Variety & AMD, VirusShare, PlayDrone  & API calls, Smali code, Methods name\\ 
 \hline

\end{tabular}}
\label{tab:dynamicTable}

\end{center}
\end{sidewaystable}
%\end{adjustbox}
%\end{landscape}

\begin{sidewaystable}
\begin{center}
\caption{A comparison of a representative set of the literature work on Dynamic analysis approaches compared detection methods, results, and limitations.}

\scalebox{0.75}{
\begin{tabular}{p{10em} p{8em} p{18em} p{30em}}
\hline
\textbf{Ref} & \textbf{Detection Method} & \textbf{Results}  & \textbf{Limitation}  \\ \hline

Sihag~\cite{SihagSV020} & Statistical  & From 260 apps, 43 apps stole sensitive data, 2 apps fetched email data, 21 apps showed ads services, 10 apps tried to jailbreak the device and 8 tried to root & Not practical for real devices, such as enabling USB debugging. \\
\hline

Bhatia \etal~\cite{BhatiaK17}  &Statistical & Accuracy: 0.88 & Strace need a rooted device, which makes applying this method dangerous on a real device.  \\ 
 \hline
Feng \etal~\cite{FengMSXM18}  & ML &Accuracy: 0.96-0.98 & MonkeyRunner and DroidBox may ignore important triggers that indicate malicious behavior of malware due to MonkeyRunne triggers only the user interface (UI) \\
 \hline
Hu \etal~\cite{HuJC20} &ML & Accuracy:0.90   &   \\
 \hline
Zhang\etal~\cite{ZhangMZRNJY22} &Fuzzy AHP  & Accuracy: 0.993&  \\
\hline
Mahindru\etal~\cite{MahindruS17} &ML  &Accuracy:   &   \\
\hline
Guerra\etal~\cite{Guerra-ManzanaresV22}  & comparison & NA  &  Limit number of samples and number of events to collect data \\
\hline
Casolare \etal~\cite{CasolareDIMMS21}  &ML  & Accuracy: 0.89 & Low accuracy compared with a static method.\\
\hline

Mahdavifar \etal~\cite{MahdavifarKFAG20} & ML \& DL &Accuracy:0.978 & Dataset proposed included only five types of malware and 17341 samples.  \\
\hline
Wit \etal~\cite{WitBH22}  & ML  & Accuracy: 0.72 &  the data collected from 47 real devices, accuracy is low  \\
\hline
Zhang \etal~\cite{ZhangLZP19}  & ML &Accuracy: 0.96-0.98  & Depends on the method names based on the API call, which can change new and updated malware \\
\hline
Onwuzurike \etal~\cite{OnwuzurikeMACRS19} & ML & F1-score: 0.99 & Abstracted API call sequences as features, it does not consider other potentially relevant features \\
\hline
\end{tabular}}
\label{tab:dynamicTable2}
\end{center}
\end{sidewaystable}


\subsection{Hybrid Approach} \hfill\\

Hybrid approaches use both dynamic and static features. The study~\cite{WangZH22} suggested a hybrid method for Android malware detection that combines static and dynamic techniques. Static analysis is used to compare the differing permission patterns of malicious and benign entities using a machine learning approach. The memory heap constructs a dynamic feature base by extracting the reference relationships between objects. The results on a real-world dataset of 21,708 applications show that the approach outperforms the well-known detector. Consequently,in the context of memory, Jang \etal introduced Andro-Dumpsys, a hybrid anti-malware system for Android malware detection. The core concept of this approach is the combination of malware-centric data with malware creator-centric data for analysis. Andro-Dumpsys employs volatile memory acquisition to extract features and behavioral characteristics from malware samples during execution. It analyzes both the malware itself (malware-centric) and attributes related to the malware creators, such as coding styles, techniques, and digital certificates (malware creator-centric). The system performs similarity matching between new malware samples and known malware samples, as well as similarity matching with the information of known malware creators. Combines the results of these two similarity detectors to make the final decision about whether a new sample is malware or not. The authors argue that incorporating malware creator information can improve malware detection and classification, as malware authors often reuse code, techniques, and patterns in different malware variants~\cite{JangKWMK16}.

Other studies focused on specific types of Android malware such as Tidke \etal that introduced the Android Botnet app to detect malicious activities and prevent them from reaching user-sensitive data. The study evaluated its approach by five participants who added dummy information to their devices. Then spam messages were sent, which included a link to download a benign app called ``Help Click.'' After installation, the app captured bank details, contacts, and password details on the participant's devices. After installing the botnet detection app on these phones, the ``Help Click'' app was detected, and botnet detection found unsafe permissions granted for other installed applications~\cite{TidkeKT18}. 

Hadiprakos \etal proposed a solution that uses the Drebin dataset to extract static features and the CICMalDroid dataset for dynamic analysis. These features were fed into machine learning and deep learning models to detect malware. Guerra-Manzanares \etal proposed a KronoDroid dataset that considered hybrid features of the Android malware dataset that included the time feature in the Android malware analysis. The dataset covered all years of Android history, from 2008 to 2020. The emulator dataset comprises 28,745 malicious applications from 209 malware families and 35,246 benign samples. The dataset for real devices included 41,382 malware samples belonging to 240 malware families and 36,755 benign applications. The dataset introduced in a structured format is the only one that provides timestamped information~\cite{Guerra-ManzanaresBN21}. Consequently, the CoDroid framework proposed in~\cite{ZhangXMZLT21} is a sequence‐based hybrid Android malware detection method that uses opcode as a static method to extract features and the system calls used in the dynamic method. The generation sequence is entered into a neural network model to classify the application as malware or benign.  Amer~\etal introduced a model to predict malicious smells. The features were extracted from static and dynamic datasets. The system and API calls were used with Word2Vec for dynamic methods to extract the similarity matrix for malware and benign apps. Then, for each matrix, the KNN was applied to cluster the matrix features. The final features were fed into the classifier model to predict malware, which achieved 0.97~\cite{AmerE22}. The summarizing and comparison of these articles are shown in tables~\ref{tab:Hybrd1} and \ref{tab:Hybrd2}.

\begin{table}
\begin{center}
\caption{Various hybrid analysis approaches in the literature, focusing on the methods used, sample sizes, dataset names, and features utilized.}

\scalebox{0.75}{
\begin{tabular}{p{7.5em} p{2em} p{5em} p{8em} p{10em} p{10em}}
\hline
\textbf{Ref} &\textbf{Year} & \textbf{Method} & \textbf{Samples} & \textbf{Dataset} & \textbf{Features}  \\ \hline

Wang \etal~\cite{WangZH22} & 2022 & Hybrid & 12364 M, 9344 B& NA & Permissions, memory heap. \\
\hline

Tidke \etal~\cite{TidkeKT18} & 2018& Hybrid & NA & One botnet & Permissions, Device data \\ 
 \hline
Guerra \etal~\cite{Guerra-ManzanaresBN21} & 2021  &  Hybrid & 41382 M, 36755 B & Drebin, VirusTotal, VirusShare, and AMD, F-droid, MARVIN, and APKMirror & Manifest file, Permissions.  \\ 
 \hline
 Zhang \etal~\cite{ZhangXMZLT21} & 2021 & Hybrid & 19K M, 11500 B & Androzoo, VirusShare, Drebin & Permissions  \\
\hline
Amer \etal \cite{AmerE22} & 2022 & Hybrid & variety & variety & System call, API call, permissions. \\
\hline
\end{tabular}}
\label{tab:Hybrd1}
\end{center}
\end{table}
%\end{adjustbox}
%\end{landscape}

\begin{table}[h]

\begin{center}
\caption{Hybrid approaches, results, and limitations.}

\scalebox{0.75}{
\begin{tabular}{p{7.5em} p{6em} p{8em} p{22em}}
\hline
\textbf{Ref} & \textbf{Detection Method} & \textbf{Results}  & \textbf{Limitation}  \\ \hline

Wang \etal~\cite{WangZH22} & ML & F1-measure: 0.975 & Memory heap can be computationally expensive compared to analyzing \\
\hline

TidkeK \etal~\cite{TidkeKT18}  & ML & NA & Used one botnet and only five participants in the experiment. \\ 
 \hline
Guerra \etal~\cite{Guerra-ManzanaresBN21}  & New dataset &NA & -.  \\
 \hline
Zhang \etal~\cite{ZhangXMZLT21} &ML & Accuracy: 0.973 &   \\
\hline

Amer \etal~\cite{AmerE22} & ML & Accuracy: 0.97 & Depends on behavioral patterns, new malware with different behaviors might go undetected. \\
\hline
\end{tabular}}
\label{tab:Hybrd2}
\end{center}
\end{table}

\section{Android Malware Datasets}
\label{SectionDatasets}

Several datasets have been used in Android malware detection research, each offering a set of features and time frames for analysis. The KronoDroid dataset~\cite{Guerra-ManzanaresBN21}, spans from 2008 to 2020 and comprises 41,382 malware samples and 36,755 benign samples. This dataset embedded hybrid features with a timestamp. Another notable dataset, MalDroid2020~\cite{MahdavifarKFAG20}, published in 2020, focused solely on malware samples from 2017 to 2018, totaling 11,598 instances in 33 families. However, it lacks benign samples and timestamp features. AndMal2020~\cite{RahaliLKTGM20}, consists of 200,000 malware and benign samples without a specific time frame mentioned, featuring 191 families and only static features. In addition, the InvesAndMal2019~\cite{TaheriKL19} and AndMal2017~\cite{LashkariKTG18} datasets cover the periods 2015 to 2017, providing 426 malware samples and 5,065 benign samples. Both datasets used a hybrid approach but lacked timestamp features. The AAGM2017 dataset~\cite{LashkariKGMG17} includes 400 malware samples and 1,500 benign samples with dynamic features without timestamps. These datasets and others are shown in the table~\ref{tabDatasets}. These are commonly used publicly available datasets for research purposes~\cite{GuerraManzanares24}.

\begin{table}[htbp]
  \center
\caption{Common used Android malware datasets}

  \renewcommand{\arraystretch}{1.5}
\linespread{0.9}
\tiny 
    \begin{tabular}{p{10em}p{2em}p{7em}p{3em}p{3em}p{5em}p{3em}p{5em}p{3em}}
    \hline
    \textbf{Ref} &  \textbf{Year} & DS Name &  \textbf{Malware} &  \textbf{Benign} &  \textbf{Time frame} &  \textbf{Families} &  \textbf{Features} &  \textbf{Tstamps} \\
     \hline
Guerra \etal ~\cite{Guerra-ManzanaresBN21}& 2021&
KronoDroid & 41382 &36755& 2008–2020 &240& Hybrid & \ding{52}\\

Mahdavifar \etal ~\cite{MahdavifarKFAG20}& 2020&MalDroid2020 &11598 &0& 2017–2018 &33& Hybrid & \ding{53}\\

Rahali \etal ~\cite{RahaliLKTGM20}& 2020&AndMal2020 &200,000 &200,000& - &191& static & \ding{53}\\

Taheri \etal ~\cite{TaheriKL19}& 2019& InvesAndMal2019 & 426 &5065& 2015–2017 &42& hybrid & \ding{53}\\

Lashkari \etal ~\cite{LashkariKTG18}& 2018&AndMal2017 & 426 &5065& 2015–2017 &42& hybrid & \ding{53}\\

Lashkari \etal ~\cite{LashkariKGMG17}& 2017& AAGM2017 &400 &1500& 2015–2016 &10& Dynamic & \ding{53}\\

Wei \etal ~\cite{WeiLROZ17}& 2017&AMD &24553&0& 2010–2016 &71& APK & \ding{53}\\

Kiss \etal ~\cite{KissLLT16}& 2016&Kharon &19 &0& 2011–2015 &19& Static & \ding{53}\\

Kadir \etal ~\cite{KadirSG15}& 2015&AndroidBot &1929 &0& 2010–2014 &14& APK & \ding{53}\\

Arp \etal ~\cite{ArpSHGR14}& 2014&Drebin & 5560 &123453& 2010–2012 &179& Static & \ding{53}\\

Zhou \etal ~\cite{ZhouJ12}& 2012&MalGenome & 1260 &0& 2010–2011 &49& APK & \ding{53}\\
   \hline
    \end{tabular}
    \label{tabDatasets}
\end{table}%


\section{Concept Drift in Malware}
In supervised machine learning, a classifier is trained to predict a target variable using a labeled dataset in particular classification tasks. Within this model, concept drift refers to the change in the relationship between input data and the target variable over time~\cite{GamaZBPB14}. 
Extensive research in the domain of Android malware detection has demonstrated the high effectiveness of machine learning techniques in identifying mobile malware. Furthermore, these studies highlight the importance of addressing concept drift in malware detection~\cite{HuMZLYL17}.


\BfPara{Understanding and analysis of concept drift} Concept drift in the machine learning model may occur due to changes in the characteristics of malware, affecting the feature space or data space drift. Previous studies have investigated these two issues to understand the problem of concept drift based on the feature and data spaces~\cite{ChenZKYCPPCW23}. Guerra-Manzanares \etal conducted a comprehensive analysis of temporal data obtained from different timestamping approaches in malware and benign applications. An approximation method was developed to compare the accuracy of the obtained data, followed by the formulation of a concept drift handling method using a classifier pool. The results highlighted the significant impact of the selection of timestamping methods on detection accuracy, especially for long time frames. It confirmed the importance of relevant temporal values in the collection of datasets. However, the study did not focus on optimizing the performance of the concept drift handling method, comparing the performance of temporal data across feature sets~\cite{Guerra-ManzanaresB22a}, while Chow \etal addressed the challenge of decreasing performance in malware classifiers due to concept drift, where malware features evolve. The paper presented a framework for performing analyses of datasets affected by concept drift to understand the root causes of concept drift, which is crucial to developing robust detection methods. Additionally, the framework is evaluated by analyzing the Transcendent dataset widely used for Android malware detection. As a result, the analysis revealed two findings: first, the performance drop is primarily due to the emergence of two malware families. Second, the evolution of certain malware families and even benign samples significantly affects the performance of the classifier~\cite{ChowKLCAP23}. Furthermore, the study used drift forensics, a new area of post-hoc analysis of drifted data, by exploring the relationship between concept drift and malware family distribution. It used explainable AI methods, automatically identifying points of interest for further analysis, demonstrating the framework's effectiveness on a large mobile malware dataset, and making the code publicly available for future research~\cite{ChowKLCAP23}. However, the literature used a different approach to detect and mitigate concept drift in the context of malware that uses machine learning approaches. This work focuses on studies that investigated concept drift in Android malware. It worthy to mention here that the attackers may use adversarial attacks to deceive the model by changing on the features of malware to appear as a benign for machine learning model~\cite{AbusnainaWAWCM23}. which introduced by Abusnaina \etal evaluated the robustness of the malware detectors against white-box and black-box adversarial attacks, which can reduce the accuracy of the detectors by up to 0.70~\cite{AbusnainaAAAJNM22}. 
 
\BfPara{\ding{172} Ensemble learning} 
Ensemble learning involves the integration of multiple prediction models to improve predictive performance. It can boost performance beyond what a single model can achieve. Three common methods are used in ensemble learning, including bagging, stacking, and boosting. Each has its unique approach to combining classification models~\cite{FernG03}. Hu \etal introduced ensemble learning with feature selection and a sliding window technique to mitigate concept drift in Android malware detection. The Naive Bayes Classifier Based on Streaming Data (NBCS) was specifically designed to handle streaming data. This method created multiple sub-classifiers, each assigned with a random feature subset. Subsequently, these sub-classifiers select features from their subsets to build individual models. Furthermore, those with low accuracy are discarded by continuously monitoring each sub-classifier's performance through a sliding window. The remaining sub-classifiers are then updated with new data from the window to adapt with concept drift, which leads to maintaining the robustness of models over time~\cite{HuMZLYL17}. Although ensemble learning often performs well, it is predominantly treated as an offline model~\cite{FernG03}. This indicates that when new features emerge, the model requires retraining, which incurs costs based on the size of the dataset. \\

\BfPara{\ding{173} Transfer learning}
Transfer learning involves using knowledge from a pre-trained model trained on a large dataset of malware samples to improve the performance of a new target model, a potentially different dataset of malware samples. Transferring the parameters of the learned model to the new model to help train the new model~\cite{FuDG21}. 
\citeauthor{GarciaDC23}  addressed the challenge of concept drift issue by using Transfer Learning (TL) techniques with specific hyperparameter configurations for malware detection. TL was used with an imbalanced dataset to identify new malware variants with a high detection rate effectively. Dynamic analysis using the cuckoo sandbox tool extracted 1135 features from APIs, signatures, and network features. The classification models used were KNN, MLP, RF, and Extreme Gradient Boosting (XGB). The performance of the model, evaluated in terms of average Matthews Correlation Coefficient (MCC), achieved an impressive 0.9775 with XGB \cite{GarciaDC23}.

\citeauthor{FuDG21} used transfer learning techniques to fine-tune an LSTM pre-trained model on new malware samples. The author constructed and trained the LSTM-based model using original benign and malware samples analyzed through static and dynamic analysis methods. Subsequently, a generative adversarial network was developed to produce augmented instances that mimic the attributes of recently surfaced malware. Some layers of the LSTM network were retrained using augmented samples, which enabled it to detect new types of malware. The study used CuckooDroid to extract the static features that included permissions and the broadcast receiver, while the dynamic features involved system calls and registered broadcast receivers at runtime. The study result predicted that malware detection achieved a classification accuracy of 0.99 when tested in augmented samples and 0.865 with malware samples in real data \cite{FuDG21}. 


 \hfill\\
\BfPara{\ding{174} Continuous/Online Learning}
 As online learning is used to solve concept drift, online models that can be dynamically updated are used. Still, they need labeled samples for updating, which can be insufficient and delayed \cite{GarciaD23}. Additionally, an imbalance in data further complicates model construction. Thus, \citeauthor{GarciaD23} suggested anomaly detection models for malware detection. Anomaly detection models, which require only benign samples for training, may be less affected by the lack of labeled malicious instances. The results indicated that the anomaly detection models outperform supervised online learning models in large data imbalances and scarcity of labels~\cite{GarciaD23}.


conducted experiments to understand the impact of feature space drift and to compare it with data space drift on the decline of malware detection models over time.  
 Their findings indicate that data-space drift primarily contributes to model declination, while feature-space drift has minimal impact. This observation was applied across various malware detectors, including Android and PE(portable executable), using different feature types and methods in various settings. The authors validate their findings using malware detection approaches based on online learning that incrementally update the feature space \cite{ChenZKYCPPCW23}. 

The automated drift adaptation techniques proposed for malware detection include DroidEvolver \cite{XuLDCX19} and its updated version, DroidEvolver++ \cite{KanPPC21}. DroidEvolver develops its model pool with five linear online learning algorithms, including Passive Aggressive (PA), Online Gradient Descent (OGD), Adaptive Regularization of Weight Vectors (AROW), Regularized Dual Averaging (RDA), and Adaptive Forward-Backward Splitting (Ada-FOBOS). DroidEvolver aims to mitigate aging across models, ensuring robust malware detection despite shared initialization datasets. During detection, DroidEvolver used weighted voting among ``young'' models to classify applications based on the API call features. A Juvenilization Indicator (JI) is utilized to identify aging models, prompting updates to both feature sets and detection models when necessary \cite{XuLDCX19}. 
The new version, DroidEvolver++, aimed to mitigate the rapid performance degradation caused by the poisoning of the model itself~\cite{KanPPC21}.

\BfPara{\ding{175} Hierarchical contrastive learning}
The work by  \citeauthor{ChenDW23}  found that after training an Android malware classifier on data for a year, the F1 score dropped from 0.99 to 0.76 within just 6 months of deployment in new test samples. The authors proposed new methods based on active learning. They select new samples for human analysts to label and then add these labeled samples to the training set to retrain the classifier. The main idea is to use similarity-based uncertainty, which the authors found to be more robust against concept drift than previous active learning approaches. The authors introduce a new hierarchical contrastive learning scheme and a sample selection technique to train the Android malware classifier~\cite{ChenDW23} continuously.

\BfPara{\ding{176} Active learning} Active learning aims to optimize the performance of the machine learning model by carefully choosing a small subset of data samples for manual annotations to reduce the cost of labeling effectively, then using these annotated samples to retrain the model and adapt it to changing data distributions \cite{AlamFMR24}. Three standard methods for sample selection in active learning include uncertainty-based, diversity-based, and expected model change. The uncertainty-based approach identifies and evaluates uncertainties of new data points, prioritizing the annotation of the least confident ones. However, the diversity-based approach selects data points that better represent the overall distribution of unlabeled samples. Finally, the expected model change approach targets samples that are expected to have the most substantial impact on the current model parameters~\cite{AlamFMR24}. Alam \etal proposed a neural network model called MORPH to mitigate concept drift in malware detection using pseudo-labels. This approach reduces manual labeling efforts compared to traditional active learning techniques while improving performance in adapting to evolving malware landscapes~\cite{AlamFMR24}. MORPH, a self-learning approach, fights against concept drift in malware detection by continuously retraining a neural network model. Initially trained on labeled data, it leverages pseudo-labels generated from its predictions on unlabeled samples for monthly retraining. A key strength of MORPH lies in its sample selection algorithm, which selects information data points (labeled and pseudo-labeled) for retraining. This reduces the need for extensive manual labeling of new data compared to traditional active learning. Although active learning also aims to minimize the effort to label, human intervention is required to select the most informative data points from the unlabeled pool. MORPH automates this process through the pseudo-labels (self-learning) method, making it potentially more efficient for adapting to evolving malware. 

\citeauthor{MolinaCoronadoMMM23} The study explored the effectiveness of retraining methods in maintaining the ability of malware detectors over time. It examined two aspects: the frequency of model retraining and the data used for retraining. The research compared periodic retraining with a concept drift detection method, triggering retraining only when necessary, and explores sampling methods like fixed-sized windows of recent data and active learning.  Their experiments indicated that the retraining approach successfully maintains the performance of existing malware detection models with static features and in an environment with concept drift \cite{MolinaCoronadoMMM23}.


\BfPara{\ding{177} Other optimization approaches} Pendlebury \etal proposed TESSERACT statistical framework based on conformal prediction theory to detect aging malware detectors during deployment before their performance declines to an unacceptable level \cite{PendleburyPJKC19}. This framework extended a new version of in \cite{BarberoPPC22}. Both investigations employed a conformal evaluator, leveraging the concept of nonconformity to detect and exclude new instances that deviate from the training distribution and are likely to be incorrectly classified. The associated applications are subsequently isolated for additional examination and labeling.

Some of the studies used dynamic features such as system call \cite{Guerra-ManzanaresLB22}. \citeauthor{Guerra-ManzanaresLB22} investigated concept drift using dynamic features system calls on imbalanced datasets. The study proposed a solution to handle concept drift, characterizing concept drift behavior, and evaluating timestamping approaches. The study used the KronoDroid dataset from 2008 to 2020 and analyzed concept drift over a continuous seven-year period \cite{Guerra-ManzanaresLB22}. Another reason that can introduce drift is the source of data collected from a real device or emulator. This problem was explored in the study \cite{Guerra-ManzanaresLB22b}. The study investigated the impact of data sources by comparing models generated from actual device and emulator data sets. In addition, it examines the effect of different timestamping options on detection performance and characterizes concept drift using a global interpretability method. The study used the KronoDroid dataset and focused on evaluating the impact of data sources and timestamps on model performance rather than optimizing detection performance.  Furthermore, it analyzed the impact of data source variation and timestamp alternatives on continuous mobile malware detection and compared learning models generated from the emulator and real device data sets for system calls \cite{Guerra-ManzanaresLB22b}. \citeauthor{RochaDCDJ23} depends on the semantics of the assembly code and proposed Asm2Vec. This technique learns vector representations from the assembly code as a potential method to prevent concept drift in malware classification. Asm2Vec can capture semantic similarities in assembly instructions to mitigate drift. The study compared machine learning models trained with features extracted using Asm2Vec embeddings. The results demonstrated that models using Asm2Vec features achieved lower performance. This indicates that Asm2Vec might not effectively capture the evolving features of malware, especially in light of adversarial changes. This finding highlights the need for further investigation into alternative methods that can more effectively address concept drift in malware classification \cite{RochaDCDJ23}.

\citeauthor{MailletM23} proposed methods to optimize neural network models for malware detection. Machine learning models for malware detection are susceptible to performance decay over time due to concept drift, where the character of malware keeps evolving. The authors propose a model-agnostic protocol that can be applied to various neural network architectures. A model-agnostic protocol refers to a set of methods that can be used for different machine learning models, regardless of their architecture. This means that the protocol does not require modifying the core neural network architecture itself and offers general methods such as feature reduction and most recent data for validation during training that were used in the study instead of randomization \cite{MailletM23}. Additionally, they introduced a new loss function, Drift-Resilient Binary Cross-Entropy, designed to be more effective against concept drift. Their research demonstrates that the enhanced model outperformed a baseline model in identifying new malware instances \cite{MailletM23}. These studies result and limitation are shown in tables \ref{tab:conceptDrift}


\begin{landscape}
\begin{table}[H]
\caption{A comparison of a related set of the
literature work on concept drift in Android malware, compared
temporal across the used method, problem, features
result and limitation. (*: Not related to Android directly)}

\renewcommand{\arraystretch}{1.5}
\linespread{0.8}
\tiny 
\begin{tabular}{p{9em} p{2em} p{10em} p{10em} p{10em} p{20em} p{20em}}
\hline
\textbf{Ref} & \textbf{Year} & \textbf{Method} & \textbf{Problem} & \textbf{Features} & \textbf{Result} & \textbf{Limitation}  \\
\hline
Hu \etal~\cite{HuMZLYL17} & 2017& Ensemble learning  & Feature selection and sliding window & Permissions, Action, API call & Naive Bayes (NB) outperformed both the baseline Hoeffding Tree (HT) and the Hoeffding Option Tree (HOT), achieving an accuracy of 0.95 &  The dataset utilized includes various datasets from undefined and discontinued time frames.

\\
\hline
% Feature-Space vs. Data-Space Drift.
Xu \etal~\cite{XuLDCX19}  & 2019 & Online learning & Identify aging models &Static, API call & Achieved F-measure 0.95 &
Models may struggle with high-dimensional and nonlinear malware data distribution. Models degrade rapidly as a result of self-poisoning  \\ 
 \hline

 Pendlebury \etal~\cite{PendleburyPJKC19} & 2019 & Optimization  &Spatial and temporal bias & & A new set of constraints  to eliminate this bias and a new metric to measure classifiers & \\ 
 \hline

 Kan \etal~\cite{KanPPC21} & 2021 & Online learning &  Poisoning model &   Static,API call  &Suggests prioritizing high-quality pseudo-labels for model updates & limitation in dataset and features  \\ 
 \hline

Guerra \etal~\cite{Guerra-ManzanaresB22a} & 2022 & Understanding & Resilient models & NA & last modification and first-seen timestamp are the best options for creating enduring ML models for Android malware detection.  &  Did not focus on
optimizing the performance of the concept drift handling.  \\ 
 \hline


 Barbero \etal~\cite{BarberoPPC22} & 2022 & Optimization  & Rejection framework &NA & guidance on optimal settings for using transcendent framework &  Flagging uncertain samples instead of making a definitive classification. \\ 
 \hline

Guerra \etal~\cite{Guerra-ManzanaresLB22} & 2022 &Optimization  &Imbalanced data  & Dynamic, system calls & F-score accuracy0.95 & Focused only system calls as a feature for detection, which is sensitive to change for the same app in a different run. \\ 
 \hline

Guerra \etal~\cite{Guerra-ManzanaresLB22b} & 2022 &  & & Dynamic, system calls & Source of data (emulator vs real device) can significantly impact the concept drift patterns & Did not consider features that do not affect when run in different devices such as static features. \\ 
 \hline

Chow \etal~\cite{ChowKLCAP23} & 2023 & Understanding & Factors cause the drift & NA& Performance declined to the emergence of only two malware families. ML model affected by new benign samples & Used a single dataset and only five families.  \\ 
 \hline

Rocha \etal~\cite{RochaDCDJ23} & 2023 &Optimization & Embeddings to solve data drift&  Static,NA &Asm2Vec does not reduce drift effects & - \\ 
 \hline
Garcia \etal~\cite{GarciaD23} & 2023 * & Online Learning &Use anomaly detection & - & Anomaly detection outperform supervised online.& - \\ 
 \hline

Maillet \etal~\cite{MailletM23} & 2023 * & Optimization &Feature reduction, loss function improvement, & - & detecting 15.2\% more malware than a baseline model. & - \\ 
 \hline

Molina \etal\cite{MolinaCoronadoMMM23} & 2023 & Active Learning & Minimizing the cost of retraining supervised models. & All in external datasets  & Average performance improvement of 0.20 compared to the original versions of the classifiers. & Need for real-world validation and potential computational costs. \\ 
 \hline

Chen \etal~\cite{ChenDW23} & 2023 & Contrastive learning &Real time updates & All in external datasets  & Reduces the false negative rate from 0.14 to 0.9, and false positive rate from 0.86 to 0.48 & 
Need for real-world validation and potential computational costs.\\
 \hline

Alam \etal~\cite{AlamFMR24} & 2024* & Active learning & Reduces manual labeling efforts when combined with active
learning & -  &  enhances previous approaches in automatic concept drift adaptation for detecting malware. & Applying in Windows malware not Android  \\ 
 \hline

\end{tabular}
\label{tab:conceptDrift}
\end{table}
\end{landscape}

\subsection{Summary and Research Gaps}
Based on the literature, there are three main approaches used to analyze and detect Android malware: static analysis, dynamic analysis, and hybrid analysis. Regardless of the type of feature used (static, dynamic, or a combination), most studies incorporating machine learning approaches have achieved good accuracy up to 0.99. Each approach has its own strength and weakness, as described in the literature.

The main gap that has been ignored in these approaches is the concept drift. Consequently, the researcher investigates this problem in the context of Android malware detection. Although the effectiveness of the proposed solutions is high, they face several significant limitations. One major issue is the quality and consistency of the datasets used in different studies. For example, Hu \etal used various datasets from undefined or discontinued time frames, leading to a lack of consistency that affects the reliability of their results~\cite{HuMZLYL17}. Additionally, the dependence on specific features, such as system calls or API calls, limits the generalizability of models. Guerra \etal focused only on system calls, which are sensitive to changes for the same app in different runs, possibly decreasing the robustness of their model \cite{Guerra-ManzanaresLB22}. Models also struggle with high-dimensional and nonlinear malware data distributions, as highlighted by~\cite{XuLDCX19}. Xu \etal~\cite{XuLDCX19} noted that model performance was massively degraded due to self-poisoning. Furthermore, the effectiveness of models is commonly measured using metrics that may not fully capture the complexities of malware detection, such as imbalanced datasets. 

Last but not least, studies like those by Molina \etal and Chen \etal who emphasized the need for real-world validation to ensure the applicability and efficiency of their findings, show that there is a significant gap in real-world validation of proposed methods~\cite{MolinaCoronadoMMM23, ChenDW23}. In addition, there is a notable lack of focus on optimizing the performance of concept drift handling. Guerra \etal identified resilience model features but did not explore methods to overcome concept drift challenges, such as including static features and forgetting models \cite{Guerra-ManzanaresLB22b}. There is also a demand for a comprehensive analysis of both static and dynamic features. The theme of existing studies focuses mainly on static or dynamic features only, as proposed in studies by Guerra \etal~\cite{Guerra-ManzanaresLB22} and Kan \etal~\cite{KanPPC21}. 

On the other hand, active learning and anomaly detection have shown promise, but these methods need a more comprehensive evaluation across different datasets with different timestamps. Lastly, there is a gap in understanding and mitigating model forgetting during retraining models. Therefore, effective strategies are needed to retain the model from previous data while adapting to new data, as continuous learning techniques suggest. Addressing these issues will help develop a more reliable and effective malware detection model. 

We noted that most research on malware detection focuses on binary classification. This approach misses the complexities involved in identifying and classifying specific malware families. There is a significant gap in addressing how different malware families evolve and how concept drift affects their detection. Furthermore, existing models often cannot adapt to new and emerging malware families in real-time. Additionally, there is an insufficient exploration of multiclassification methods that could improve the detail and accuracy of malware detection by identifying specific families rather than just malware or benign.


\chapter{Research Methodology}
\label{Chapter4}

In the rapidly growing landscape of cybersecurity, the detection and mitigation of Android malware detection remain essential challenges. With the accumulation of machine learning models in malware detection, researchers and practitioners also face the persistent issue of concept drift---the phenomenon where statistical properties or defining features of the target variable change over time in an unpredictable manner. Addressing concept drift is critical for maintaining the efficacy and reliability of malware detection models over time.

This research aims to investigate the prevalence of concept drift in different machine learning models used for Android malware detection, to understand the root cause of concept drift, and to develop effective strategies for detecting and mitigating this phenomenon. By analyzing factors contributing to concept drift, identifying suitable metrics and methods for its detection, and developing retraining approaches to adapt machine learning models to the changes in the data distributions. In summary, this research seeks to improve the resilience of malware detection based on machine learning models.

\section{Research Philosophy}
This research combines both research philosophies, relying on positivism to understand the problem of prevalence concept drift and pragmatism to find solutions by investigating concept drift in Android malware detection models. 

\BfPara{Positivism} Underscores the use of empirical observation and scientific methods to find objective facts, which can be practical for understanding the prevalence and severity of concept drift within Android malware detection. This philosophical view provides a strong framework for testing the existence and features of concept drift within the domain of Android malware detection. Quantitative methods enable the objective measurement and analysis of concept drift phenomena. Additionally, it can offer an evaluation of the efficacy of proposed methods for automated concept drift detection and retraining strategies through defined metrics. 

\BfPara{Pragmatism} Emphasizing the practical application of research findings to address real-world problems. The dynamic nature of concept drift encourages researchers to adapt and develop methods to solve this problem. This perspective is particularly valuable for developing practical solutions and strategies for mitigating concept drift in Android malware detection models, taking into account the various contexts and limitations faced by researchers in this domain.

By combining positivism and pragmatism, this research can benefit from the powers of both philosophical viewpoints. It can utilize strict scientific methods to investigate the phenomenon of concept drift while also prioritizing and evaluating practical solutions that can be implemented in real-world settings. This integrated approach can address both theoretical issues and practical challenges in the field of Android malware detection.


\section{Research Design}
This research will use both quantitative and qualitative methods to investigate concept drift in Android malware detection models. This approach will provide an understanding of the factors that cause the prevalence of concept drift, the methods to detect it, retraining strategies, and addressing model forgetting in the context of Android malware detection. 

\subsection{Concept Drift: Understanding and Adaption Strategies} We will follow a qualitative research method that includes data collection through a systemic review of the literature and conduct case studies and data analysis, which we elaborate as the following: 

\BfPara{Data Collection} 
The first phase of this study involves conducting a comprehensive \textbf{survey} (systematic literature review) of existing research on concept drift in Android malware detection models using machine learning and deep learning techniques. This survey seeks to explore the existing body of knowledge about the underlying causes of concept drift and the different adaptation approaches used in this domain. In addition, the survey will aim to identify and explain the research challenges inherent in this domain. Through this study,  the results will shed light on the factors contributing to concept drift and model forgetting, thus establishing a basis for further research.

\BfPara{Case Study and Data Analysis}
The second phase will be a case study, through the identification and examination of several rapidly evolving malware families (both first seen and last update), focusing on the particular change in their behavior or code (static and dynamic features) that leads to concept drift. The comparative analysis will be applied to the case study by comparing different versions of malware families and investigating the changes in features that result in concept drift in machine learning models. To do that, we intend to use the existing dataset \cite{Guerra-ManzanaresBN21} that included more than 41K malware samples classified into 240 families and 428 static and dynamic features. Then we will collect new malware samples that have the same families to study the drift in the experiments. To collect malware samples, there are repositories such as MalwareBazaar~\cite{MalwareB12:online}.

\subsection{Factors Causing the Drift in Data and Features space}
The datasets used for Android malware detection include a set of features that are classified into static and dynamic features. 

To understand how these features are impacted by malware's evolving characteristics, leading to concept drift, we will conduct a detailed analysis. This analysis will focus on identifying which specific features(static or dynamic) are most susceptible to changes over time and result in concept drift.
For static features, we will investigate how different features, such as newly introduced APIs, changes in permission models, and modifications in code obfuscation techniques by adversaries affect the detection models. This will include investigating historical data to specify when and how these features have changed and correlating these shifts with changes in detection accuracy. Furthermore, in relation to {\bf RQ4.2}, we can examine and quantify the effect of deprecated features of the Android operating system on the concept drift. 

For dynamic features, we will analyze how changes in runtime behaviors, such as changes in the system call sequences, impact the models. Furthermore, we intend to understand the types of concept drift (e.g., abrupt, gradual, incremental, recurring) that mostly affect each type of feature, so we can better develop our detection models to adapt to these changes.

\subsection{Evaluation Metrics and Performance Monitoring}

\BfPara{Metrics Selection}
We will focus on establishing robust evaluation metrics and continuous monitoring strategies to evaluate the performance of machine learning models used to detect Android malware, especially in datasets with unbalanced distributions.
In malware datasets, the distribution of families tends to be unbalanced. To be able to efficiently measure the performance of detecting families, the measure should handle imbalances such as AUC and F1-scores.


\BfPara{Continuous Model Monitoring} 
To establish a system for continuous evaluation of the model as new data of malware becomes available. We can create a feedback loop for the model that includes the following:

\begin{itemize}
    \item[\ding{172}] {\bf Real-time Data Ingestion.}  Develop techniques to periodically include new malware samples into the dataset, ensuring that the model is tested using up-to-date new malware samples. For example, integrated threat intelligence repositories with machine learning models 

      \item[\ding{173}] {\bf Model Re-evaluation.} based on the ingested data, the model will be tested aging these samples. As a result, we will reevaluate the model performance to reflect any changes in the data landscape.
      
        \item[\ding{174}] {\bf Model Adjustment.} Based on the evaluation results, adjust model parameters or retrain the model using updated datasets to better cope with new types of malware that may be emerging. Thus, by evaluating and measuring the change in the model performance when using up to date samples, and the impact of model adjustment step to up to date model training we can answer the \bf{RQ3}.

\end{itemize}

\subsection{Concept Drift Detection}
Previous studies focused on detecting concept drifts in machine learning models that utilized binary classification tasks (i.e., malware or benign). In this work, our objective is to investigate the detection of concept drift within the malware family. It is essential to distinguish the families of malware to develop a mitigation strategy and risk assessment for specific malware. For example, the \textit{Adware} type includes the following families ``ewind'' and ``shedun'', which is different from the \textit{Ransomware} type that includes families like ``lockscreen'', ``slocker'', and ``smsspy''. 

To do that, we have different datasets, only one of them supports the timestamp. But this dataset determined the malware family without including malware types, as shown in section \ref{SectionDatasets}. So, we need to generate a new dataset collected from different sources. To validate the concept drift in various machine and deep learning models mostly used in the literature.  

\BfPara{Feature engineering and data analysis}
Extract the appropriate characteristics from the datasets. This may include techniques such as feature selection, dimensionality reduction, and normalization to prepare the data for analysis. In the concept drift, it is important to highlight two critical points. The first one deals with the balance of data, especially for malware families. The second issue will use NLP techniques to find a semantic correlation between the old and new malware families. 

\BfPara{Retraining strategies}
The adapted retraining strategy should ensure the minimization of the cost of the model retraining process while preserving the semantic function to identify the relations between the existing and new malware sample to be classified into the target family or a new family, indicating a zero-day attack. In addition, to maintain model accuracy, we will investigate various continuous and reinforcement learning solutions.


\subsection{Model Immunity Against Forgetting}
While the concept drift solution focuses on the updated model to overcome the outdated trained model. Model retraining can result in another issue called model forgetting, which is also known as catastrophic forgetting, where a machine learning model loses previously obtained knowledge when it is retrained on new data. In other words, model forgetting occurs when a detection model, after it was retrained to identify new malware, loses the ability of the model to effectively identify older malware that it was previously able to detect.

To address model forgetting in the context of retraining machine learning models for Android malware detection, we will follow the following steps: 

\begin{itemize}

\item[\ding{172}] Gather datasets that contain benign and Android malware applications, in addition to static, and dynamic features, malware families, and support timestamp. While there is only one data set---KronoDroid~\cite{Guerra-ManzanaresBN21}---that supports timestamp, the malware families are not balanced. We intend to explore alternative methods for approximating timestamps and balance malware families, such as grouping by collection date or version number and annotated based on the threat intelligence reports and malware signatures. This step can address {\bf RQ4.1}

\item[\ding{173}] Establish a baseline model by training the model on a representative malware dataset and evaluating the performance using selected metrics such as AUC and F1-score.

\item[\ding{174}] Segment the dataset chronologically to simulate the evolving malware and retrain the model periodically with these new data segments.

\item[\ding{175}] Previous studies employed various retraining strategies such as ensemble learning, online learning, and active learning. To add these strategies, we will examine reinforcement learning to mitigate forgetting. Reinforcement learning will be employed to automate and adjust the parameters of the model based on feedback from performance metrics, to keep previously learned knowledge. The performance of retrained models will be evaluated on both actual and new validation datasets. Additionally, We will enhance retraining algorithms and combine hybrid approaches to maximize model immunity against forgetting.   

    \end{itemize}
 

% \subsection{Timeline}
% \begin{enumerate}
%     \item  Create a comprehensive systematic review of the literature (3-4 months). (Publish the result)
%     \item Prepare dataset that will support the malware family and type (2-3 months).
%     \item Validating machine learning and deep learning models to detect concept drift and investigate in forgetting issue (3-4 months). ( Publish the result). 
%     \item Propose a solution to mitigate concept drift issues based on best retrain strategy, opportunity reinforcement learning (3-4 months). ( Publish the result)
%     \item Writing final dissertation 
    
%\end{enumerate}

\subsection{Conclusion}
Android security is gaining importance due to the widespread use of mobile for everyday tasks, the large user base, and the criticality of some of the applications used by mobile users make them an attractive target. Android malware detection models face significant challenges due to a phenomenon known as concept drift. This phenomenon is considered a critical problem that frequently struggles to adapt to the dynamic nature of malware. This research aims to investigate the prevalence of concept drift in various machine learning models for Android malware detection and develop effective strategies to understand, analyze, detect, and mitigate concept drift to provide sustained performance and robust detection models over time. Additionally, it is important to investigate the model of forgetting after applying the retaining strategies to be sure that the mode enhances the detection of new samples while preserving the ability detection old samples.









 
%\include{Chapters/Chapter5} 

%----------------------------------------------------------------------------------------
%	THESIS CONTENT - APPENDICES
%----------------------------------------------------------------------------------------

\appendix % Cue to tell LaTeX that the following "chapters" are Appendices

% Include the appendices of the thesis as separate files from the Appendices folder
% Uncomment the lines as you write the Appendices

%\include{Appendices/AppendixA}
%\include{Appendices/AppendixB}
%\include{Appendices/AppendixC}


%----------------------------------------------------------------------------------------
%	BIBLIOGRAPHY
%----------------------------------------------------------------------------------------

%to cite, use \textcite{citation name}, if there is no cite, the bibliography will not be generated.
% \printbibliography[
% heading=bibintoc,
% title={References}
% ] 
%bibintoc to include ref. list in toc, title for references.

%----------------------------------------------------------------------------------------

% {\footnotesize
\bibliography{reference}
%}
% \bibliography{}

\end{document}  
